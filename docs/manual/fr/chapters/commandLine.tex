\chapter{Lancement de XLogo en ligne de comandes}
\noindent
Voici la syntaxe de la commande à taper pour lancer XLogo:
\begin{center}
 \texttt{java -jar xlogo.jar [-a] [-lang fr] [-memory 64][fichier1.lgo fichier2.lgo ...] } 
\end{center}
Détails des différentes options disponibles: \\
\begin{itemize}
 \item Attribut \texttt{-lang}: cet attribut permet de spécifier un langage particulier pour XLogo. Ce paramètre écrase celui contenu dans le fichier personnel de configuration nommé \texttt{.xlogo}. Les différentes langues sont accessibles suivant ce tableau: \\
\begin{center}
\begin{tabular}{|c|c|c|c|c|c|c|c|c|}
\hline
Français & Anglais & Espagnol & Allemand & Arabe & Portugais & Espéranto & Galicien& Grec \\
\hline
fr & en & es & de & ar & pt & eo & gl&el\\
\hline
\end{tabular}
\end{center}
\vspace{0.5cm}
\item Attribut \texttt{-a}: cet attribut permet d'exécuter dès l'ouverture de XLogo la commande principale contenue dans les fichiers chargés au démarrage.\\
\item Attribut \texttt{-memory}: cet attribut permet de fixer la mémoire allouée à la machine virtuelle Java.\\
\item fichier1.lgo, fichier2.lgo ...: ces fichiers au format \texttt{.lgo} sont chargés au démarrage de XLogo. Ces fichiers peuvent être locaux ou distants c'est à dire que leur adresse peut aussi bien désigner l'arborescence locale qu'une adresse Internet.\\
\item Attribut \texttt{tcp\_port}: cet attribut permet de sélectionner un numéro de port précis pour les communications réseaux. Le port par défaut est 1948. Voir p.\pageref{reseau}
\end{itemize}
\vspace{0.5cm}
Quelques exemples de commandes: \\
\begin{itemize}
 \item \texttt{java -jar xlogo.jar -lang es prog.lgo}: Les fichiers \texttt{xlogo.jar} et \texttt{prog.lgo} sont contenus dans le répertoire courant. Cette commande lance XLogo configuré en espagnol et charge ensuite le fichier \texttt{prog.lgo} (Qui doit par conséquent être rédigé en espagnol...)\\
 \item \texttt{java -jar xlogo.jar -a -lang en http://xlogo.tuxfamily.org/prog.lgo}:\\
 Cette commande lance XLogo configuré en anglais et charge le fichier nommé \\ \texttt{http://xlogo.tuxfamily.org/prog.lgo}. Pour finir, la commande principale définie dans ce fichier est exécutée au démarrage.\\
\end{itemize}


