\chapter{Lancer Xlogo depuis le WEB}
\noindent
Vous disposez d'un site web sur lequel vous parlez de XLogo. Mieux encore, vous souhaiteriez mettre à disposition certains des programmes que vous avez créer. Plutôt que de distribuer simplement les fichiers \texttt{.lgo}, il serait plus agréable pour l'utilisateur de pouvoir lancer XLogo en ligne afin de tester directement ces exemples. Voici la procédure à suivre:\\ \\
Le lancement de Xlogo en ligne est assuré par la technologie \textsc{JAVA WEB START}. En fait, il suffit de fournir sur votre site un lien vers un fichier d'extension \texttt{.jnlp}, cela assurera l'exécution de XLogo.\\ \\
\textbf{Création du fichier d'extension \texttt{jnlp}}\\ \\
Voici ci-dessous un exemple d'un tel fichier. Ce fichier est en fait celui utilisé dans la rubrique \og exemples\fg\ du site français. Il permet de charger le programme traçant le dé dans la section 3D. Les grandes lignes d'explication sont données par la suite.
\begin{verbatim}

 <?xml version="1.0" encoding="utf-8"?>
<jnlp spec="1.5+" codebase="http://downloads.tuxfamily.org/xlogo/common/webstart">
<information>
  <title>XLogo</title>
  <vendor>xlogo.tuxfamily.org</vendor>
  <homepage href="http://xlogo.tuxfamily.org"/>
  <description>Logo Programming Language</description>
  <offline-allowed/>
</information>

<security>
	<all-permissions/>
</security>

<resources>
  <j2se version="1.4+"/>
  <jar href="xlogo.jar"/>
</resources>

<application-desc main-class="Lanceur">
  <argument>-lang</argument>
  <argument>fr</argument>
  <argument>-a</argument>
  <argument>http://xlogo.tuxfamily.org/fr/html/examples-fr/3d/de.lgo</argument>
</application-desc>
</jnlp>

\end{verbatim}
Ce fichier est écrit en respectant le format XML. La partie importante est située à la fin du fichier notamment ces 4 lignes:
\begin{verbatim}

  <argument>-lang</argument>
  <argument>fr</argument>
  <argument>-a</argument>
  <argument>http://xlogo.tuxfamily.org/fr/html/examples-fr/3d/de.lgo</argument>

\end{verbatim}
C'est ici que sont spécifiés les paramètres de lancement. 
\begin{itemize}
 \item Les deux premières lignes force l'utilisation de la langue française.
\item La dernière ligne indique l'adresse du fichier à charger.
\item La troisième ligne indique que la commande principale de ce fichier sera exécutée au démarrage de XLogo.
\end{itemize}
\vspace{0.5cm}
\textbf{Dernière petite astuce}: Si vous désirez ne pas surcharger le serveur de Tuxfamily, vous avez la possibilité de déposer le fichier \texttt{xlogo.jar} sur votre site. Pour lier le fichier \texttt{.jnlp} à ce fichier, il vous suffit de changer l'adresse contenue à la ligne 2 après \texttt{codebase=}
