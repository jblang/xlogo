\chapter*{Introduction}
Le Logo est un langage qui a été développé dans les années 60 par Seymour Papert. Celui-ci s'appuyait sur une théorie originale de l'apprentissage, appelée le constructionisme, dont on peut résumer le concept en anglais par l'expression \og \textit{learning-by-doing}\fg. \\

  Le langage \logo\ permet de développer incontestablement certaines compétences mathématiques et logiques, c'est un excellent langage pour débuter avec la programmation et apprendre les rudiments tels que les boucles, les tests, les procédures...  L'utilisateur peut déplacer un objet appelé « tortue » sur l'écran à l'aide de commandes aussi simples que \texttt{avance}, \texttt{recule}, \texttt{tournedroite} et autres. A chaque déplacement, la tortue laisse un trait derrière elle et ainsi on peut créer des dessins. Le fait de pouvoir donner les instructions dans un langage quasiment usuel facilite grandement l'apprentissage. Une utilisation plus avancée est aussi possible, on peut ainsi manipuler des objets tels que les listes, les mots ou encore les fichiers.\\

\logo\ est un langage interprété: c'est à dire que les commandes écrites par l'utilisateur sont exécutées immédiatement par l'ordinateur. On se rend compte directement lors du déroulement du programme des erreurs commises ce qui favorise l'apprentisage.\\

\xlogo\ est donc un interpréteur du langage \logo. L'adresse du site principal de l'application est:
\begin{center}
\texttt{http://xlogo.tuxfamily.org/}
\end{center}
Vous pourrez télécharger le logiciel et la documentation. Une gallerie de multiples exemples permet de mieux cerner les aptitudes du logiciel.\\

\xlogo\ supporte actuellement 10 langues (anglais, allemand, arabe, asturien, espagnol, espéranto,français, galicien, grec et portugais) et a été écrit en \textsc{Java}. Ce langage présente l'avantage d'être multiplateforme c'est à dire que l'application \xlogo\ tournera indépendamment du système d'exploitation installé. Que vous soyez sous Linux, sous Windows ou encore sous MAC, pas de problèmes, la petite tortue s'offre à vous!\\

\noindent \textbf{\xlogo\  est placé sous licence GPL:} \\ \\
C'est donc un logiciel libre ce qui garantit à l'utilisateur:
\begin{enumerate}
 \item la liberté d'exécuter le logiciel, pour n'importe quel usage ;
 \item la liberté d'étudier le fonctionnement d'un programme et de l'adapter à ses besoins, ce qui passe par l'accès aux codes sources ;
 \item la liberté de redistribuer des copies ;
 \item la liberté d'améliorer le programme et de rendre publiques les modifications afin que l'ensemble de la communauté en bénéficie.
\end{enumerate}
\noindent \textbf{Structure du manuel:}\\ \\
Ce manuel va vous permettre de découvrir \xlogo.
\begin{itemize}
 \item La première partie sera consacrée à la description de l'interface et des différents menus.
 \item Ensuite, une série de chapitres vous présentera les premières instructions de base de \xlogo. La difficulté dans l'enchaînement des notions se veut graduée. Des exercices d'application sont proposés en fin de chapitre, vous trouverez leurs corrigés en annexe.
\item Enfin, une série de thèmes particuliers est abordée pour les utilisateurs avancés.
\item En annexe, vous trouverez notamment la description de la totalité des primitives ainsi que les différentes options de lancement de \xlogo.
\end{itemize}
\vspace{0.5cm}
Ce manuel est disponbile sous divers formats:
\begin{itemize}
 \item \textsc{PDF}: http://downloads.tuxfamily.org/xlogo/downloads-fr/manual-fr.pdf
 \item \textsc{HTML zippé}: http://downloads.tuxfamily.org/xlogo/downloads-fr/manual-html-fr.zip
 \item \LaTeXe: Source du manuel: http://downloads.tuxfamily.org/xlogo/downloads-fr/manual-src-fr.zip
 \item \textsc{JavaHelp}: Via le menu Aide-Manuel en ligne de \xlogo
\end{itemize}
