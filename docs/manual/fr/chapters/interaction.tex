\chapter{Programme interactif}
{ }\hfill\textbf{Niveau:} débutant
\section{Communiquer avec l'utilisateur}
\noindent Nous allons réaliser un petit programme qui demande à l'utilisateur son nom, son prénom et son age. A la fin du questionnaire, le programme répond par un récapitulatif su style:
\begin{verbatim}
Ton nom est:........
Ton prénom est: .......
Ton age est: .......
Tu es mineur ou majeur
\end{verbatim}
\noindent \textsc{Pour cela, nous allons utiliser les primitives suivantes:}  \\
\begin{itemize}
\item \texttt{lis}:\hspace{4cm}  \textcolor{red}{ \texttt{lis [Quel est ton age? ] "a}}\\
Affiche une boîte de dialogue ayant pour titre le texte contenu dans la liste (ici, \og Quel est ton age?\fg). La réponse donnée par l'utilisateur est mémorisée sous forme d'un mot ou d'une liste (si l'utilisateur tape plusieurs mots) dans la variable \texttt{:a}.\\
\item \texttt{donne}:\hspace{4cm}  \textcolor{red}{ \texttt{donne "a 30}}\\ \\
Donne la valeur 30 à la variable \texttt{:a}\\
\item \texttt{phrase, ph}:\hspace{4cm}  \textcolor{red}{ \texttt{phrase [30 k] "a }}\\ \\
Rajoute une valeur dans une liste. Si cette valeur est une liste, assemble les deux listes.\\ 
\begin{verbatim}
phrase [30 k] "a ---> [30 k a]
phrase [1 2 3] 4 ---> [1 2 3 4]
phrase [1 2 3] [4 5 6] ---> [1 2 3 4 5 6]

\end{verbatim} 
\end{itemize}
Nous obtenons le code suivant:
\begin{verbatim}
pour question
lis [Quel est ton age?] "age
lis [Quel est ton nom?] "nom
lis [Quel est ton prénom?] "prenom
ecris phrase [Ton nom est: ] :nom
ecris phrase [Ton prénom est: ] :prenom
ecris phrase [Ton age est: ] :age
si ou :age>18 :age=18 [ecris [Tu es majeur]] [ecris [Tu es mineur]]
fin
\end{verbatim}
\section{Programmer un petit jeu.}
\noindent \textsc{L'objectif de ce paragraphe est de créer le jeu suivant:}\\ \\
Le programme choisit un nombre au hasard entre 0 et 32 et le mémorise. Une boîte de dialogue s'ouvre et demande à l'utilisateur de rentrer un nombre. Si le nombre proposé est égal au nombre mémorisé, il affiche \og gagné \fg dans la zone de texte. Dans le cas contraire, le programme indique si le nombre mémorisé est plus petit ou plus grand que le nombre proposé par l'utilisateur puis rouvre la boîte de dialogue. Le programme se termine quand l'utilisateur a trouvé le nombre mémorisé.\\ \\
Vous aurez besoin d'utiliser la primitive suivante:\\
\texttt{hasard}: \hspace{4cm} \textcolor{red}{\texttt{hasard 8}} \\
\texttt{hasard 20} rend donc un nombre choisi au hasard entre 0 et 19.\\
Rend un nombre au hasard compris entre 0 et 8 strictement.\\ \\
\textsc{Voici quelques règles à respecter pour réaliser ce petit jeu:}
\begin{itemize}
\item Le nombre mémorisé par l'ordinateur sera mémorisé dans une variable nommée \texttt{nombre}.
\item La boîte de dialogue aura pour titre: \og Propose un nombre: \fg.
\item Le nombre proposé par l'utilisateur sera enregistré dans une variable nommée \texttt{essai}.
\item La procédure qui permet de lancer le jeu s'appellera  \texttt{jeu}.
\end{itemize}
\vspace{0.5cm}
\noindent \textsc{Quelques améliorations possibles:} \\
\begin{itemize}
\item Afficher le nombre de coups.
\item Le nombre recherché devra être compris entre 0 et 2000.
\item Vérifier si ce que rentre l'utilisateur est réellement un nombre. Pour cela, utiliser la primitive \texttt{nombre?}. \\
Exemples: \begin{tabular}[t]{l}
\texttt{nombre? 8} est vrai.\\
\texttt{nombre? [5 6 7]} est faux. ([5 6 7] est une liste et non pas un nombre)\\
\texttt{nombre? "abcde} est faux. ("abcde est un mot et non pas un nombre)
\end{tabular}
\end{itemize}