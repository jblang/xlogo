\chapter*{Enkonduko}

Logo estas programlingvo disvolvata en la jaroj 60 de Seymour Papert.
Li apogis sin sur originala teorio pri la lernado, nomata konstruismo,
kies koncepton oni povas resumi per la esprimo
\og\textit{lerni-per-fari}\fg.

La lingvo \logo\ ebligas vaste disvolvi iujn matematikajn kaj logikajn
kapablojn; ^gi estas bonega lingvo por ekstudi la programadon kaj
lerni la bazojn kiel la buklojn, la provojn, la procedurojn...  La
uzulo povas movi objekton nomatan «testudo» sur la ekrano per komandoj
tiel simplaj kiel \texttt{anta^uen}, \texttt{malanta^uen},
\texttt{dekstren} kaj aliaj.  Post ^ciu movo, la testudo lasas ^spuron
malanta^u si kaj tiel oni povas krei desegnojn.  La fakto povi ordoni
en lingvo preska^u kutima faciligas multe la lernadon.  Anka^u pli
altnivela uzado eblas; oni povas manipuli objektojn tiajn kiel
listojn, vortojn a^u e^c dosierojn.

\logo\ estas lingvo interpretata, tio estas, la komandoj skribitaj de
la uzulo estas tuj rulotaj de la komputilo.  Oni rimarkas rekte de la
rulado de la programo, la erarojn faritajn; tio favoras la lernadon.

\xlogo\ estas do interpretilo por lingvo \logo.  La adreso de la ^cefa
loko de la programo estas:
\begin{center}
\texttt{http://xlogo.tuxfamily.org/}
\end{center}
Vi povos de^suti la programon kaj la dokumentaron.  Galerio de kelkaj
ekzemploj ebligas pli bone ekkoni la kapablojn de la programo.

\xlogo\ subtenas nun 10 lingvojn (angla, araba, astura, esperanto,
germana, hispana, franca, galega, greka kaj portugala) kaj estas
verkita en \textsc{Java}.  Tiu programlingvo havas la avanta^gon esti
plurplatforma, tio estas, ke la programo \xlogo\ ruli^gos sendepende
de la mastruma sistemo instalita.  ^Cu vi estas en GNU/Linukso, en
Vindozo a^u e^c en Makinto^so, ne estas problemo; la malgranda testudo
sin oferas al vi!\\

\noindent \textbf{\xlogo\ estas sub permesilo GPL:} \\ \\
^Gi estas do libera programo; tio garantias al la uzulo:
\begin{enumerate}
 \item la liberecon ruli la programon, por ia ajn celo;
 \item la liberecon studi la funkciadon de programo kaj adapti ^gin al
   siaj bezonoj; tio postulas alireblon al la fontokodojn;
 \item la liberecon disdoni kopiojn;
 \item la liberecon plibonigi la programon kaj publikigi la modifojn
   por ke la tuta komunumo profitu.
\end{enumerate}
\noindent \textbf{Strukturo de la gvidlibro:}\\ \\
Tiu gvidlibro ebligos vin ekkoni \xlogo n.
\begin{itemize}
\item La unua parto estas dedi^cita al priskribo de la interfaco kaj
  de la diversaj menuoj.
\item Poste, kelkaj ^capitroj al vi prezentas la unuajn bazajn
  instrukciojn de \xlogo.  La malfacileco de la en^cenado de la nocioj
  estas gradita.  Ekzercoj aplikaj estas proponataj je la fino de
  ^capitro; iliaj korektigoj estas en krom^capitro.
\item Finfine, kelkajn specialajn temojn oni traktas por la altnivelaj
  uzuloj.
\item En krom^capitro, vi trovos la priskribon de ^ciuj primitivoj,
  kaj la diversajn elekta^jojn por ekruli \xlogo n.
\end{itemize}
\vspace{0.5cm}
^Ci tiu gvidlibro haveblas en diversaj formatoj:
\begin{itemize}
 \item \textsc{PDF}: http://downloads.tuxfamily.org/xlogo/downloads-eo/manual-eo.pdf
 \item \textsc{HTML zipita}: http://downloads.tuxfamily.org/xlogo/downloads-eo/manual-html-eo.zip
 \item \LaTeXe: Fontokodo de la gvidlibro: http://downloads.tuxfamily.org/xlogo/downloads-eo/manual-src-eo.zip
 \item \textsc{JavaHelp}: Per la menuo Helpo-Gvido dumrule de \xlogo
\end{itemize}
