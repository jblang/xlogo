\chapter{Oftaj demandoj -- Konsiloj}

\section{Se mi forvi^sas proceduron en la redaktilo, ^gi reaperas
  ^ciam!}

Kiam oni eliras el la redaktilo, tiu limigas sin konservi a^u
^gisdatigi la enhavon de la redaktilo.  La sola rimedo forvi^si
proceduron en \xlogo{} estas uzi la primitivon \texttt{nomon\_vi^su}
a^u \texttt{nv}.

Ekzemple: \texttt{nv "toto} $\longrightarrow$ forvi^su la proceduron
\texttt{toto}.

\section{Mi uzas la esperantan version sed mi ne povas skribi la
  ^capelitajn signojn!}

Dum vi tajpas en la komandlinio a^u la redaktilo, se vi premas la
dekstran musbutonon, aperos ekmenuon.  En tiu menu, aperas la
tradiciaj redaktagoj (kopiu/enpo^sigu, fortran^cu, algluu/elpo^sigu)
kaj la ^capelitaj signoj de l' Esperanto, kiam tiun lingvon oni
elektis.

\section{En la langeto sono de la dialogfenestro Agordaj iloj, neniu instumento haveblas.}

Kalkafoje, la listo de MIDI-instrumentoj ne aperas en \texttt{Agordaj
  iloj / Sono} kaj oni ne povas uzi ^ciel la funkciojn sonajn de
\xlogo{}.  Adresu vin al:
\begin{center}
 \texttt{http://java.sun.com/products/java-media/sound/soundbanks.html}
\end{center}
De^sutu unu el la sonbenkoj (soundbank) proponitaj (minimal, midsize
a^u deluxe), poste maldensigu ^gin en \texttt{C:\textbackslash Program
  Files\textbackslash Java\textbackslash jre1.6.0\_05\textbackslash
  lib\textbackslash audio\textbackslash}.

\begin{itemize}
\item La dosiero \texttt{jre1.6.0\_05} respondas al via versio de la
  instalita JRE.
 \item Se la dosiero \texttt{audio} ne ekzistas, necesos krei ^gin.
 \item Necesos alinomi la maldensigitan dosieron en:
   \texttt{soundbank.gm}
\end{itemize}
\vspace{0.2cm}
Poste rerulu \xlogo{} kaj iru do rigardi en \texttt{Agordaj iloj / Elektebloj / Sono}

\section{Kiel faru por tajpi rapide komandon jam uzitan?}
\begin{itemize}
\item Unua metodo: per la muso, klaku en la historiejo sur la dezirata
  linio; ^gi reaperos tuj en la komandlinio.
\item Dua metodo: per la klavaro, la sagoj supren kaj malsupren
  ebligas navigi en la listo de la laste tajpitaj komandoj.
\end{itemize}

\section{Kiel oni povas helpi vin?}

\begin{itemize}
\item Raportante pri cimoj (eraroj) konstatitaj.  Estus e^c pli bone,
  se vi kapablus sisteme aperigi konstatitan problemon.
\item Viaj sugestoj por la plibonigo, estas ^ciam bonvenaj.
\item Helpante pri tradukoj.
\item Malgranda kura^gigo ^ciam bonfaras!
\end{itemize}
