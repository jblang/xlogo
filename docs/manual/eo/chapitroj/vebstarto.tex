\chapter{Ekruli \xlogo{} disde la reto}

Vi havas retpa^gon sur kiu vi parolas pri \xlogo{}.  E^c pli bone: vi
deziras havigi iujn programojn kiujn vi verkis.  Anstata^u simple
distribui la dosierojn \texttt{.lgo}, estus pli agrable por l' uzulo
povi ruli \xlogo{} enrete por provi rekti tiujn ekzemplojn.  Jen la
sekvenda proceduro:

La enretan ruleblon de \xlogo{} certigas la te^hnologio \textsc{JAVA
  WEB START}.  Efektive, sufi^cas meti en vian retpa^gon ligilon al
dosiero kun fina^jo \texttt{.jnlp}; tio certigas la ruladon de
\xlogo{}.

\subsubsection*{Krei dosieron kun ligilo \texttt{jnlp}}

Jen ekzemplo de tia dosiero.  Tiu dosiero estas efektive tiu uzata en
la sekcio \og exemples\fg\ de la franca retpa^go.  ^Gi ebligas ^sargi
la programon grafikantan la ludkubon en la sekcio pri 3D.  La grandaj
linioj por klarigi aperas poste.

\begin{verbatim}
<?xml version="1.0" encoding="utf-8"?>
<jnlp spec="1.5+" codebase="http://downloads.tuxfamily.org/xlogo/common/webstart">
<information>
  <title>XLogo</title>
  <vendor>xlogo.tuxfamily.org</vendor>
  <homepage href="http://xlogo.tuxfamily.org"/>
  <description>Logo Programming Language</description>
  <offline-allowed/>
</information>

<security>
 <all-permissions/>
</security>

<resources>
  <j2se version="1.4+"/>
  <jar href="xlogo.jar"/>
</resources>

<application-desc main-class="Lanceur">
  <argument>-lang</argument>
  <argument>fr</argument>
  <argument>-a</argument>
  <argument>http://xlogo.tuxfamily.org/fr/html/examples-fr/3d/de.lgo</argument>
</application-desc>
</jnlp>
\end{verbatim}

Tiu dosiero estas skribita observante la formaton XML.
La grava parto estas je la fino, ^cefe tiuj $4$ linioj:
\begin{verbatim}
  <argument>-lang</argument>
  <argument>fr</argument>
  <argument>-a</argument>
  <argument>http://xlogo.tuxfamily.org/fr/html/examples-fr/3d/de.lgo</argument>
\end{verbatim}

Ja tie oni indikas la ekrulajn parametrojn.
\begin{itemize}
\item La du unuaj linioj devigas uzi la francan lingvon.
\item La lasta linio indikas la adreson de la ^sargota dosiero.
\item La tria linio indikas ke la startiga komando de tiu dosiero
  estas rulota je la starto de \xlogo{}.
\end{itemize}
\vspace{0.5cm} \textbf{Lasta konsileto}: Se vi deziras ne tro^sar^gi
la servilon de Tuxfamily, vi povas meti la dosieron \texttt{xlogo.jar}
sur vian servilon.  Por ligi la dosieron \texttt{.jnlp} al tiu
dosiero, sufi^cus ^san^gi l' adreson en la dua linio, post
\texttt{codebase=}.
