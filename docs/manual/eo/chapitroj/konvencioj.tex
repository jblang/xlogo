\chapter{Konvencioj adoptitaj en XLOGO}
Jen prezentado de iuj aferoj pri la programlingvo LOGO mem kaj de
aliaj pri XLOGO specife.

\section{Komandoj kaj interpretado}
Programlingvo LOGO konsistas el internaj komandoj: tiajn komandojn oni
nomas \textbf{primitivoj}.  ^Ciu primitivo atendas iun nombron de
parametroj nomataj \textbf{argumentoj}.  Por ekzemplo, la primitivo
\texttt{ev} kiu ebligas vi^si l' ekranon prenas nul argumenton, dum la
primitivo \texttt{sum} atendas du argumentojn: \texttt{ sum 2 3}
skribos 5 redone.

Estas tri specoj de argumentoj en LOGO:
\begin{itemize}
\item \textbf{La nombroj:} Iuj primitivoj atendas nombrojn kiel
  argumenton.  Ekzemple \texttt{anta^uen 100}
\item \textbf{La vortoj:} ^Ciuj vortoj komenci^gas per ".  Ekzemplo de
  primitivo kapabla labori pri vortoj estas la primitivo
  \texttt{skribu}.
\begin{center}
\texttt{skribu "saluton} 
\end{center}
Tiu komando ka^uzas l' aperon de la vorto \texttt{saluton} en la
teksta areo.

Rimarku ke se vi forgesas la ", l' interpretilo respondos per
erarmesa^go.  Efektive, \texttt{skribu} atendas argumenton, sed por l'
interpretilo \texttt{saluton} signifas nenion, ^car ^gi estas nek
nombro nek vorto nek listo nek jam difinita proceduro.
\item\textbf{La listoj:} Ilin oni difinas inter rektaj krampoj.
\end{itemize}
\vspace{0.5cm} \textbf{Rimarku:} La nombroj estas traktataj jen kiel
nombraj valoroj, jen kiel vortoj.  Ekzemple: \texttt{skribu unuan 12}
redonas 1.  Iuj primitivoj akceptas ^generalan formon, tio estas, ili
povas ricevi nedifinitan nombron de argumentoj.  Jen la listo de tiuj
primitivoj:
\begin{center}
  \begin{tabular}{cccc}
    \texttt{skribu} & \texttt{sumon}&\texttt{produton} &\texttt{a^u}\\
    \hline
    \texttt{kaj}&\texttt{liston}&\texttt{frazon}& \texttt{vorton}\\
  \end{tabular} 
\end{center}
Por sciigi l' interpretilon ke oni uzos ilin sub ilia ^generalan
formon, oni tajpu la komandon inter rondaj krampoj; jen kelkaj
ekzemploj:
\begin{verbatim}
skribu (sumon 1 2 3 4 5)
15

(list [a b] 1 [c d])
Kiel uzi [[a b] 1 [c d]]?

se (kaj 1=1 2=2 8=5+3) [an 100 dn 90]
\end{verbatim}

\section{Proceduroj}
Krom tiuj primitivoj, vi povas difini viajn proprajn komandojn.  Oni
nomas ilin \textit{proceduroj}.  La procedurojn oni komencas difini
per helpo de la vorto \texttt{por} kaj oni finas difini per la vorto
\texttt{fino}.  Oni uzas la proceduran redaktilon internan je XLOGO
por tajpi ilin.  Jen malgrandan ekzemplon:
\begin{verbatim}

por kvadrato 
ripetu 4 [antaŭen 100 dekstren 90]
fino

\end{verbatim}

Anka^u tiaj proceduroj rajtas akcepti argumentojn.  Por tio, oni uzas
variablojn.  Variablo estas vorto al kiu oni povas rilatigi valoron.
Jen tre simpla ekzemplo:

\begin{verbatim}

por tuto :a :b
skribu sum :a :b
fino

tuto 2 3 -----> 5

\end{verbatim}

\section{La speciala signo  \og\textbackslash\fg}
La signo \og \textbackslash \fg \ (maloblikva streko) ebligas krei
vortojn enhavantajn spacojn a^u enhavantajn linisalton.  \og
\textbackslash n\fg \ enmetas linisalton kaj \og
\textbackslash\textvisiblespace\fg \ enmetas spacon en vorton.
Ekzemple:
\begin{verbatim}
skribu "xlogo\ xlogo
xlogo xlogo
skribu "xlogo\nxlogo
xlogo
xlogo
\end{verbatim}
Tial por skribi signon \og \textbackslash\fg \ oni tajpu ^gin duoble:
\og \textbackslash\textbackslash\fg.

Same, la signoj \og ( ) [ ] \# \fg\ estas limiloj de la lingvo Logo
kiuj ne povas esti uzataj en vortoj.  Oni povos enmeti ilin per aldoni
signon \og \textbackslash \fg\ anta^ue.

\textbf{^Ciu signo \og \textbackslash \fg \ sola estos ignorita.  Tio
  tre gravas specife por administri dosierojn.}

Por establi la aktualan dosierujon je \texttt{C:\textbackslash Miaj
  dokumentoj}, necesos tajpi:
\begin{verbatim}
dosierujon_provizu "c:\\Miaj\ dokumentoj
\end{verbatim}
Rimarku l' uzadon de \og \textbackslash\textvisiblespace \fg \ por
indiki la spacon inter \og Miaj\fg \ kaj \og dokumentoj\fg.  Se
aliflanke, vi ne metas la duoblan maloblikvan strekon, la vojo
difinitas estos tiam \texttt{c:Miaj dokumentoj} kaj la interpretilo
skribos erarmesa^gon.

\section{Reguloj pri uskleco}

\xlogo{} ne diferencas uskle pri la nomoj de proceduroj kaj
primitivoj.  Tial, pri la proceduro \texttt{kvadrato} difinita
anta^ue, ^cu vi tajpus \texttt{KVADRATO}, ^cu \texttt{KvaDRato}, l'
interpretilo de komandoj ^guste interpretos kaj rulos
\texttt{kvadrato}.  Male, \xlogo{} diferencas en listoj kaj vortoj:
\begin{verbatim}
skribu "Saluton ----> "Saluton (oni konservas la majusklan S)
\end{verbatim}
\section{Operatoroj kaj sintakso}
Estas du manieroj skribi kelkajn komandojn.  Ekzemple, por adicii 4
kaj 7, estas du ebloj:
\begin{itemize}
\item jen oni uzas la primitivon \texttt{sumon} kiu atendas du
  argumentojn: oni skribas \texttt{sumon 4 7 }
\item jen oni uzas l' operatoron +: oni skribas \texttt{4+7}.
\end{itemize}
La du havas saman efikon.  Jen la listo de rilatoj inter operatoroj
kaj primitivoj:
\begin{center}
  \begin{tabular}{|c|c|c|c|}
    \hline
    \texttt{sumon} & \texttt{subtrahon} & \texttt{produton} & \texttt{dividon}\\
    \hline
    + & - & * & / \\
    \hline
    \texttt{a^u} & \texttt{kaj}&\texttt{egala?}& \\
    \hline
    | & \& &=&\\
    \hline
  \end{tabular}\end{center}
\vspace{0.25cm}
Ekzistas anka^u du operatoroj de numeraj provoj rilataj al neniu primitivo:
\begin{itemize}
\item Operatoro \og malpli granda a^u egala\fg{} \texttt{<=}
\item Operatoro \og pli granda a^u egala\fg{} \texttt{>=}
\end{itemize}

\textbf{Atentu:} Neniu spaco inter la signoj \verb+>+ kaj \verb+=+!

\textbf{Rimarku:} La du operatoroj | et \& estas specifaj operatoroj
de XLOGO.  Ili ne ekzistas en la tradiciaj versioj de LOGO.  Jen
kelkaj ekzemploj de uzo:
\begin{verbatim}
s 3+4=7-1 ----> vera
s 3=4 | 7>=49/7 ----> vera
s 3=4 & 7=49/7 ----> malvera
\end{verbatim}
