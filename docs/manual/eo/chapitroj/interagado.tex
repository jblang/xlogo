\chapter{Interaktiva programado}
\label{interaktiva}

{ }\hfill\textbf{Nivelo:} komencanto
\section{Komuniki kun l' uzulo}

Ni realigos malgrandan programon kiu demandas de l' uzulo ^slian
nomon, baptonomon kaj a^gon.  Je l' fin' de la demandaro, la programo
respondos per memorigilo jene:
\begin{verbatim}
Via familinomo estas:........
Via baptonomo estas: .......
Via aĝo estas: .......
Vi estas (mal)plenkreskulo
\end{verbatim}
\textsc{Por tio, ni uzos la jenajn primitivojn:}
\begin{itemize}
\item \texttt{legu}:\hspace{4cm}  \textcolor{red}{ \texttt{legu [Kiom estas via a^go? ] "a}}

  Aperigas dialogfenestron kun titolo kiel la listo argumento (tie,
  \og Kiom estas via a^go?\fg). La respondo donita de l' uzulo estas
  memorita kiel vorto a^u listo (se l' uzul' tajpas plurajn
  vortojn) en la variablo \texttt{:a}.

\item \texttt{provizu, p}:\hspace{4cm}  \textcolor{red}{ \texttt{provizu "a 30}}

  Donas la valoron $30$ al la variablo \texttt{:a}

\item \texttt{frazon, fr}:\hspace{4cm}  \textcolor{red}{ \texttt{frazon [30 k] "a }}

  Aldonas valoron en liston.  Se tiu valoro estas listo, kunigas la du listojn.

\begin{verbatim}
frazon [30 k] "a ---> [30 k a]
frazon [1 2 3] 4 ---> [1 2 3 4]
frazon [1 2 3] [4 5 6] ---> [1 2 3 4 5 6]
\end{verbatim} 
\end{itemize}
Jen la kodo:
\begin{verbatim}
por demandaro
legu [Kiom aĝas vi?] "aĝo
legu [Kio estas via familinomo?] "famnomo
legu [Kio estas via baptonmo?] "bapnomo
skribu frazon [Via familinomo estas: ] :famnom
skribu frazon [Via baptonomo: ] :bapnomo
skribu frazon [Via aĝo estas: ] :aĝo
se aŭ :aĝo>18 :aĝo=18 [skribu [Vi estas plenkreskulo]] [skribu [Vi estas malplenkreskulo]]
fino
\end{verbatim}

\section{Programi malgrandan ludon}

\textsc{La celo de ^ci tiu sekcio estas krei la jenan ludon:}

La programo elektas hazardan nombron inter $0$ kaj $32$ kaj memoras
^gin.  Dialogfenestro aperas kaj demandas l' uzulon enigi nombron.  Se
la proponita nomo estas egala al la memorita nomo, ^gi skribas \og
venkis\fg en la tekstejo.  En mala okazo, la programo indikas ^cu la
nombro memorita estas pli malgranda a^u granda ol la nombro proponita
de l' uzulo; poste ^gi reaperigas la dialogfenestron.  La programo
haltos kiam l' uzulo trovas la memoritan nombron.

Vi bezonos uzi la jenan primitivon:

\texttt{hazardon, hzd}: \hspace{4cm} \textcolor{red}{\texttt{hazardon 8}} 

\texttt{hazardon 20} donas nombron hazarde elektitan inter $0$ kaj $19$.

\textsc{Jen kelkaj reguloj respektendaj por realigi tiun ludon:}
\begin{itemize}
\item La nombro memorita de l' komputilo estas memorata en variablo
  nomata \texttt{nombro}.
\item La dialogfenestro havos por titolo; \og Proponu nombron:\fg.
\item La nombro proponita de l' uzulo estos registrita en variablo
  nomata \texttt{provo}.
\item La proceduro kiu ebligas ruli la ludon nomi^gos  \texttt{ludo}.
\end{itemize}

\vspace{0.5cm}
\textsc{Kelkaj eblaj plibonigoj:}
\begin{itemize}
\item Skribi la nombro de provoj.
\item La nombro ser^cota estu inter $0$ kaj $2000$.
\item Konstati ^cu tio enigita de l' uzulo estas vere nombro.  Por
  tio, uzu la primitivon \texttt{nombra?}.

  Exemples: \begin{tabular}[t]{l}
    \texttt{nombra? 8} estas vera.\\
    \texttt{nombra? [5 6 7]} estas malvera ([5 6 7] estas listo sed ne
    nombro).\\
    \texttt{nombra? "abcde} estas malvera ("abcde estas
    vorto sed ne nombro).
\end{tabular}
\end{itemize}
