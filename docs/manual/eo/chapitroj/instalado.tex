\chapter{Instalado de \xlogo}
\noindent 
\begin{itemize}
 \item Unue, vi bezonas instali rulmedion JAVA en via komputilo.  Iru al tiu pa^go:
\begin{center}
 \texttt{http://java.sun.com/javase/downloads/index.jsp}
\end{center}
De^sutu la JRE (Java Runtime Environment) korespondantan al via
mastruma sistemo (Vindozo, GNU/Linukso...); poste instalu ^gin.
\item Due, necesas de^suti la dosieron \texttt{xlogo.jar} estanta ^ce
  la adreso:
\begin{center}
 \texttt{http://downloads.tuxfamily.org/xlogo/common/xlogo.jar}
\end{center}
Se ne, pli simple, iru al la loko de \xlogo, ^ce la adreso
\texttt{http://xlogo.tuxfamily.org}; poste elektu la lingvon kaj la
menuon de^suti.
\end{itemize}
\section{Agordado de \xlogo}
\subsection{Medio GNU/Linukso}
En Ubuntu 8.04:
\begin{enumerate}
 \item Por instali JAVA:
\begin{itemize}
\item Sistemo -> Administri -> Administrilo de pakoj Synaptic
\item Instali la pakon \texttt{sun-java6-jre}
\end{itemize}
\item Por malfermi la dosieron \texttt{xlogo.jar} per duobla klako:
\begin{itemize}
\item Dekstreklaku sur \texttt{xlogo.jar}, Atributoj
\item Tabo \og Malfermi per\fg: Elektu Sun Java Runtime 
\end{itemize}
 \item Asociigu la dosiertipon \texttt{lgo} al \xlogo:
\begin{itemize}
 \item Dekstreklaku sur \texttt{xlogo.jar}, Atributoj
 \item Tabo \og Malferm per\fg:
 \item Butono \og Aldoni\fg
 \item En \og Uzi komandon personigitan:\fg, tajpu:
\begin{center}
\texttt{java -jar vojo\_al\_xlogo.jar} 
\end{center}
\end{itemize}
\end{enumerate}
\textbf{Rimarku:} \xlogo\ estas enhavita en la distribuo OpenSuse.
\subsection{Medio Windows}
Komence, se vi duobleklakas sur la ikono de \xlogo, la programo devas
starti.  Se ^gi okazas, iru al la sekva paragrafo.  Se ne, la ka^uzo
estas ke alia programo okupi^gas pri la dosierojn de tipo
\og\texttt{jar}\fg\ (ofte, malkunpremaj programoj, kiel WinZip kaj
aliaj).

Jen kiel asociigi la programon \og\texttt{java}\fg\ al la dosieroj de
tipo \og \texttt{jar}\fg.  (Kelkaj vojoj povas esti malsamaj, la^u ke
vi posedas Vindozo 98, 2000, XP...)
\begin{enumerate}
\item Starto –> Parametroj —> Elekto de dosieroj...
\item Klaku poste sur la tabo \og Dosiertipoj\fg\ (la
  3\textsuperscript{a}).
\item Ser^cu en la listo la elekta^jojn rilatajn al dosieroj JAR
  (Dosieroj JAR, Ruldosieroj JAR, Ar^hivo JAR...)
\item Elektu tiun dosiertipon kaj klaku sur \og Modifi...\fg 
\item Nova fenestro aperas, tiam elektu \og Modifi... \fg 
\item Elekt tiam \og Traser^ci...\fg
\item Necesas indiki la vojon al \texttt{javaw.exe}, ekzemple
\begin{center}
 \texttt{c:\textbackslash Program Files\textbackslash java\textbackslash j2re1.4.1\textbackslash bin\textbackslash javaw.exe}
\end{center}
\item Tion farinte, aperas en la kampo Aplika^jo uzata por efektivigi
  la agon:
\begin{center}
 \texttt{c:\textbackslash Program Files\textbackslash java\textbackslash j2re1.4.1\textbackslash bin\textbackslash javaw.exe}
\end{center}
Necesas tiam aldoni ^ce la fino:
\begin{center}
\texttt{"c:\textbackslash Program Files\textbackslash java\textbackslash j2re1.4.1\textbackslash bin\textbackslash javaw.exe" -jar "\%1" \%*}
\end{center} 
(Rimarku ke necesas spaceto je ^ciu flanko de -jar)
\item Poste, nur fermu ^ciun fenestron kaj poste duobleklaku sur la
  ikono de \xlogo.
\end{enumerate}
Se tio ne ^ciam funkcias, estas dua eblo: Vi malfermu konsolon MSDOS
(Starto --> Programoj -–> Komandoj MSDOS a^u Starto --> Programoj -->
Iloj --> Invito MSDOS); poste tajpu la ordonon jenan:
\begin{center}
 \texttt{java -jar la\_vojo\_kie\_trovi^gas\_la\_dosiero}
\end{center}
Por ekzemplo: \texttt{java -jar c:\textbackslash xlogo\textbackslash xlogo.jar}\\ \\
Se tio enuas vin, sisteme devi tajpi tiun ordonon, tajpu tion en
teksta dosiero kaj konservu ^gin ekzemple sub la nomo
\texttt{xlogo.bat}.  Nur restas duobleklaki sur \texttt{xlogo.bat} por
startigi \xlogo n.

\subsubsection*{Asociigi la dosierojn de tipo \texttt{lgo} kun \xlogo}
Mi ne atingis agordi tion en Vista.  (Sed mi ne tre ser^cis...
Rimarku, amatoroj!  Dankon pro komuniki al mi la solvon.)

Principe, la dosieroj de tipo \texttt{.lgo} ne estas rekonataj de via
komputilo; kiam vi duobleklakas ilin, dialogskatolo aperas por demandi
al vi kiun aplika^jon oni uzu por malfermi la dosieron.
\begin{itemize}
\item Indiku \og Alia\fg; poste indiku la vojon al la programo
  \texttt{javaw.exe}
\begin{center}
^Generale, \texttt{c:\textbackslash Program Files\textbackslash java\textbackslash j2re1.4.1\textbackslash bin\textbackslash javaw.exe}
\end{center}
\item Doni nomon por nomi la dosierojn je tipo \texttt{lgo}.\\
Por ekzemplo: Dosieroj Logo
\item Starto -> Parametroj -> Agorda^joj de la dosieroj
\item Tabo \og Dosiertipoj \fg
\item Ser^cu en la listo la dosierojn \texttt{lgo}
\item Elektu tiun dosiertipojn; poste klaku sur \og Modifi\fg
\item Nova fenestro aperas; refoje \og Modifi\fg
\item En la kampo \og Aplika^jo uzata por efektivigi la agon\fg,
\begin{center}
\texttt{"c:\textbackslash Program Files\textbackslash java\textbackslash j2re1.4.1\textbackslash bin\textbackslash javaw.exe" -jar xlogo.jar "\%1" \%*}
\end{center} 
\item Fermu la fenestrojn.
\end{itemize}
\section{^Gisdatigoj}
\begin{center}
\includegraphics{bildoj/rss.png} \hspace{1cm} \texttt{http://xlogo.tuxfamily.org/rss.xml}
\end{center}
Por ^gisdatigi \xlogo, sufi^cas anstata^uigi la dosieron
\texttt{xlogo.jar} per ^gia nova versio.  Se vi deziras esti avertata
de la apero de ^ciu nova versio, a^u de ^ciu plibonigo, eblas aboni al
la RSS-fadeno de \xlogo.  La adreso de la RSS-fadeno estas:
\begin{center}
 \texttt{http://xlogo.tuxfamily.org/rss.xml}
\end{center}
Ekzistas pluraj softvoj ebligantaj sekvi la fadenojn RSS; se vi ne
konas tiun te^hnikon, la plej simpla estas uzi Mozilla Thunderbird:
\begin{itemize}
 \item Menu' Redakti - Parametroj de la kontoj
 \item Buton' \og Aldoni konton\fg
 \item \og Nova^joj RSS kaj blogoj\fg
 \item Nomo de la konto: \og Fadenoj RSS\fg\ por ekzemplo
 \item Butonoj \og Sekva\fg\ kaj \og Fini\fg
 \item En la fenestro \og Parametroj de la kontoj\fg, elektu tiam \og
   Fadenoj RSS\fg\ en la menu' maldekstra; poste klaku sur la butono
   \og Administri la abonoj\fg.
 \item Buton' \og Aldoni\fg
	\begin{itemize}
 	\item URL de la fadeno: \texttt{http://xlogo.tuxfamily.org/rss.xml}
	\item Aktivigu la skatolon \og Afi^si la resumon de l'
          artikolo anstata^u de^suti la retpa^gon\fg
	\end{itemize}
\end{itemize}
\vspace*{0.2cm} Jen, per la butono \og Sendi-Ricevi\fg, vi ricevos la
nova^jojn de \xlogo\ sammaniere kiel vi ricevas viajn retpo^sta^join.
\section{Malinstalado}\label{fichier_perso}
Por malinstali \xlogo, sufi^cas forigi la dosieron \texttt{xlogo.jar}
kaj la startan dosieron \texttt{.xlogo} (^gi estas lokita en via uzula
dosierujo, tio estas, \texttt{/home/via\_konto} por la gnulinuksistoj
a^u \texttt{c:\textbackslash windows\textbackslash.xlogo} por la
vindozistoj.
