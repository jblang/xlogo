\chapter{Ekruli XLogo en komandlinio}

Jen la sintakso de la komando tajpenda por ekruli XLogo:
\begin{center}
  \texttt{java -jar xlogo.jar [-a] [-lang eo] [-memory
    64][dosiero1.lgo dosiero2.lgo ...] }
\end{center}
Jen detaloj de la diversaj elektebloj:
\begin{itemize}
\item Elekteblo \texttt{-lang}: ^gi ebligas indiki homan lingvon por
  XLogo.  Tiu parametro superregas tiun enhavatan en la persona agorda
  dosiero nomata \texttt{.xlogo}.  La lingvojn oni povas elekti la^u
  tiu tabelo:
  \begin{center}
    \begin{tabular}{|c|c|c|c|c|c|c|c|c|}
      \hline
      Franca & Angla & Hispana & Germana & Araba & Portugala & Esperanto & Galega & Greka \\
      \hline
      fr & en & es & de & ar & pt & eo & gl & el \\
      \hline
    \end{tabular}
  \end{center}
  \vspace{0.5cm}
\item Elekteblo \texttt{-a}: ^gi ebligas ruli ekde la malfermo de
  XLogo la ^cefan komandon enhavatan en la dosierojn ^sargitajn je la
  starto.
\item Elekteblo \texttt{-memory}: ^gi ebligas establi la memoron
  rezervita por la virtuala ma^sino Java.
\item dosiero1.lgo, dosiero2.lgo ...: tiuj dosieroj kun fina^jo
  \texttt{.lgo} estas ^sargataj je la starto de XLogo.  Tiuj dosieroj
  povas esti lokaj a^u foraj, tio estas, ilia adreso povas indiki
  vojon en la loka hierar^hia arbo de dosierujoj a^u interretan
  adreson.
\item Elekteblo \texttt{tcp\_port}: ^gi ebligas elekti pordan numeron
  por la reta komunikado.  Apriora pordo estas $1948$.  Rigardu
  p.~\pageref{reseau}.
\end{itemize}
\vspace{0.5cm}
Jen ekzemploj de komandoj:
\begin{itemize}
\item \texttt{java -jar xlogo.jar -lang es prog.lgo}: La dosieroj
  \texttt{xlogo.jar} kaj \texttt{prog.lgo} estas en la nuna dosierujo.
  Tiu komando ekrulas \xlogo{} agordita en la hispana kaj ^sargas tuj
  poste la dosieron \texttt{prog.lgo} (kiu do devas esti redaktita en
  la hispana...).
\item \texttt{java -jar xlogo.jar -a -lang en
    http://xlogo.tuxfamily.org/prog.lgo}: Tiu komando rulas \xlogo{}
  agordita en la angla kaj ^sargas la dosieron nomatan
  \texttt{http://xlogo.tuxfamily.org/prog.lgo}.  Por fini, la ^cefa
  (startiga) komando difinita en tiu dosiero estas rulota je la
  starto.
\end{itemize}


