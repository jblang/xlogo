\chapter{FAQ - Tricks Things to know}
\section{Though I erase a procedure from the editor, it keeps on popping back!}
When you leave the editor, it just saves or updates whatever the editor contains. The only way to erase a procedure in \xlogo\ is to use the primitive \texttt{eraseprocedure} or \texttt{erp}.\\
Exemple: \texttt{erp "toto} $\longrightarrow$ erases the procedure \texttt{toto}.
 \section{I'm using the version in Esperanto but I can't write with the special characters!}
When you type in the command line or the editor, if you click with the right button, a rolling screen appears. In this menu, you can find the traditional editing functions (cut, copy, paste) and the esperanto special characters when this language is selected.
 \section{In the Sound tab from the Preferences dialogue box, no instrument can be found.}
Sometimes, the instruments list doesn't appear in \texttt{Tools/Preferences/Sound}. Go here: 
\begin{center}
 \texttt{http://java.sun.com/products/java-media/sound/soundbanks.html}
\end{center}
Download on eof the sound banks: minimal, midsize ou deluxe and uncompress it in \texttt{C:\textbackslash Program Files\textbackslash Java\textbackslash jre1.6.0\_05\textbackslash lib\textbackslash audio\textbackslash}.\\
\begin{itemize}
 \item The folder  \texttt{jre1.6.0\_05} corresponds to your installed JRE version.
 \item If the folder \texttt{audio} doesn't exist, create it.
 \item You have to rename the uncompressed file as: \texttt{soundbank.gm}
\end{itemize}
\vspace{0.2cm}
\noindent Then, restart \xlogo\ and have a look at \texttt{Tools/Preferences/Sound}
\section{How to quickly retype a command used previously?}
\begin{itemize}
\item First method: with the mouse, click on the line in the history area, it will reappear immediately on the control line.
\item Second method: with the keyboard, the Up and Down arrows allow navigation of the list of previous commands that have been typed, (very practical).
\end{itemize}
 \section{How can you help?}
\begin{itemize}
\item By reporting any observed bug. If you are able to reproduce systematically an observed problem, it is even better.
\item Any suggestion to improve the program is welcome.
\item By helping to translate. 
\item A little moral support is always welcome!
\end{itemize}