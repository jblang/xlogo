\chapter{Conventions adopted by \xlogo}

This section sets out some key points about the LOGO language itself,
and about \xlogo\ specifically.


\section{Commands and their interpretation}

The LOGO language allows certain events to be triggered by internal
commands - these commands are called \textit{primitives}. Each primitive
may have a certain number of parameters which are called \textit{arguments}.
For example, the primitive \texttt{cs}, which clears the screen,
takes no arguments, while the primitive \texttt{sum} takes two arguments.\\
\\
 \texttt{print sum 2 3} will return 5.\\
\\
 LOGO arguments are of three kinds:

\begin{itemize}
\item \textbf{Numbers:} some primitives expect numbers as an argument: \texttt{fd
100} is an example. 
\item \textbf{Words:} Words are marked by an initial \char`\"{}. An example
of a primitive which can take a word argument is \texttt{print}. 
\begin{center}
\texttt{print\ \char`\"{}hello}
\end{center}
This command displays \texttt{hello}. If you forget the
\char`\"{}, the interpreter will return an error message. In effect,
\texttt{print} expects an argument, or for the interpreter, \texttt{hello}
does not represent anything, since it is not a number, a word, a list,
or an already defined procedure.
\item \textbf{Lists:} these are defined between brackets.
\end{itemize}
\textbf{Note}: Numbers are treated in some instances as a numeric value (eg: \texttt{fd
100}), and in others as a word (eg: \texttt{print first 12} writes \texttt{1}). \\ \\

Several primitives have a general form, it means they could be used with an undefined number of arguments. All those primitives are on the table below:
\begin{center}
 \begin{tabular}{cccc}
 \texttt{print} & \texttt{sum}&\texttt{product} &\texttt{or}\\
\hline
\texttt{and}&\texttt{list}&\texttt{sentence}& \texttt{word}\\
 \end{tabular} 
\end{center}
To notify the interpreter that these primitives will be used in their general form, we have to write our command into parenthesis, look at those examples below: \pagebreak
 \begin{verbatim}
 print (sum 1 2 3 4 5)
15

(list [a b] 1 [c d])
I don't know what to do with [[a b] 1 [c d]]?

if (and 1=1 2=2 8=5+3) [fd 100 rt 90]
\end{verbatim}

\section{Procedures}

In addition to these primitives, you can define your own commands.
These are called \textit{procedures}. Procedures are introduced by
the word \texttt{to} and conclude with the word \texttt{end}. They
can be created using the internal \xlogo\ procedure editor. Here is
a short example: \begin{verbatim}

to square
repeat 4[forward 100 right 90]
end

\end{verbatim}

These procedures can take advantage of arguments as well. To do that,
variables are used. A variable is a word to which a value can be assigned.
Here is a very simple example:
\begin{verbatim}
to total :a :b
print sum :a :b
end

total 2 3
5
\end{verbatim} 
\section{Specific character \textbackslash}
The specific character \textbackslash \ (backslash) allows the creation of words containing blank or line feed symbols. If \textbackslash n is used, the phrase skips to the following line, and \textbackslash\textvisiblespace\ followed by a blank means a blank in a word.
Example:
\begin{verbatim}
pr "xlogo\ xlogo
xlogo xlogo
pr "xlogo\nxlogo
xlogo
xlogo
\end{verbatim}

\noindent You can therefore only write the \textbackslash \ symbol by typing \textbackslash\textbackslash.\\ \\
Similar behaviour, characters ( ) [ ] \#  are specific delimiters of Logo. If you want to use them in a word, you just have to add the character \textbackslash before. \\ \\
\textbf{All \textbackslash \ only symbols are ignored. This remark is especially important for the use of files.\\}
\\
 To set your current directory path to \texttt{c:\textbackslash My Documents}:
\begin{verbatim}
setdir "c:\\My\ Documents.
\end{verbatim}

Please note the use of \textbackslash\textvisiblespace \ to notify the space between My and Documents. If, you forget the double backslash, the path that will be defined will then be \texttt{c:My Documents} and the interpretor will send you an error message.

\section{Case-sensitivity}

\xlogo\ makes no distinctions on case as regards procedure names and
primitives. Thus, with the procedure \texttt{square}  as defined earlier,
whether you type \texttt{SQUARE} or \texttt{sQuaRe}, the command interpreter
will translate it correctly and execute \texttt{square}. On the other
hand, \xlogo\ is case-sensitive on lists and words: \\
 \begin{verbatim}
print "Hello ----> "Hello (the initial capital H is retained)
\end{verbatim}


\section{Operators and syntax}

There are two ways to write certain commands. For example, to add
4 and 7, there are two possibilities: you can either use the primitive
\texttt{sum} which expects two arguments: \texttt{sum 4 7}, or
you can use the operator +: \texttt{4+7}. Both have the same effect.\\
 This table shows the relationship between operators and primitives:\\
\begin{center}
\begin{tabular}{|c|c|c|c|}
\hline 
\texttt{sum}&
 \texttt{difference}&
 \texttt{product }&
 \texttt{quotient}\\
\hline
+ &
 - &
 {*} &
 / \\
\hline
\texttt{or}&
 \texttt{and}&
\texttt{equal?}&
\\
\hline
| (ALT GR+6) &
 \&&
=&
 \\
\hline
\end{tabular}             \end{center}
\vspace{0.25cm}
There are two other operators with no associated primitive:\begin{itemize}
 \item Operator ``Less than or equal to'': \texttt{<=}
\item Opérator ``Greater than or equal to'': \texttt{>=}
\end{itemize}
Note: The two operators | and \& are specific to \xlogo. They do not exist
in traditional versions of LOGO. Here are some examples of usage:

\begin{verbatim}
pr 3+4=7-1 ----> false
pr 3=4 | 7<=49/7 ----> true
pr 3=4 & 7=49/7 ----> false
\end{verbatim}