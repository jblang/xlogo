\chapter{Install \xlogo}
\noindent \begin{itemize}
 \item First of all, you will have to install the Java Runtime Environment on your computer. Go to this page:
\begin{center}
\texttt{ http://java.sun.com/javase/downloads/index.jsp}
\end{center}
Download the JRE (Java Runtime Environment) which corresponds to your operating system (Windows, Linux ...), and install it. 
\item You have to download the file \texttt{xlogo.jar} at the address: 
\begin{center}
	\texttt{http://xlogo.tuxfamily.org/common/xlogo.jar}
\end{center}
Otherwise, you can go on the \xlogo\ website, at the address \texttt{http://xlogo.tuxfamily.org}, choose english language and then click on the downloads menu.
\end{itemize}
\section{\xlogo\ Configuration}
\subsection{Linux Environment}
Under Ubuntu 8.04:
\begin{enumerate}
 \item To install JAVA:
\begin{itemize}
 \item System -> Administration -> Synaptic Package Manager
 \item Install the package \texttt{sun-java6-jre}
\end{itemize}
 \item  To open the file \texttt{xlogo.jar} double-clicking:
\begin{itemize}
 \item Right click on \texttt{xlogo.jar}, Properties
 \item Tab ``Open With'': Choose Sun Java 6 Runtime 
\end{itemize}
 \item To associate extensions \texttt{lgo} to \xlogo:
\begin{itemize}
 \item Right click on \texttt{xlogo.jar}, Properties
 \item Tab ``Open With'' 
 \item Button ``Add''
 \item Field ``Use a custom command'', write:
\begin{center}
\texttt{java -jar path\_to\_xlogo.jar} 
\end{center}
\end{itemize}
\end{enumerate}
\textbf{Note:} \xlogo\ is included in distribution OpenSuse.
\subsection{Windows Environment}
In theory, if you double-click on the \xlogo\ icon, the program should launch.
If this is the case, go on to the next section.  If not, and another application is launched instead (something like winzip, perhaps), this is because .jar files are in fact .zip files, and these are themselves executable (ie a program can be launched by clicking on them).  If your computer opens a program like winzip, it is because from its point of view files with a .jar extension can only be opened with that program.  You therefore have to deactivate the association of that program with .jar files.  To do that, follow these steps for Windows XP (some paths may differ depending on the flavour of Windows you are running, and you will have to adjust them):

\begin{enumerate}
\item Start -> Control Panel -> Switch to Classic View  -> Folder options
\item Click on the tab File Types (the third tab)
\item Find in the list of registered file types those connected with jar files (jar files, executable jar files, jar archive, etc)
\item Click the file type, and then click Advanced...
\item A new window will appear: click on Open, and then Edit...
\item Click on Browse... and navigate to javaw.exe; this is usually
\begin{center}
\texttt{c:\textbackslash{}Program Files\textbackslash{}java\textbackslash{}j2re1.4.1\textbackslash{}bin\textbackslash{}javaw.exe}
\end{center}
\item The path {}``c:\textbackslash{}Program Files\textbackslash{}java\textbackslash{}j2re1.4.1\textbackslash{}bin\textbackslash{}javaw.exe'' will then appear in the field \textit{Application used to perform action:}.  You need to make an addition to the end of this, so that it reads:
\begin{center}
\texttt{ "c:\textbackslash{}Program Files\textbackslash{}java\textbackslash{}j2re1.4.1\textbackslash{}bin\textbackslash{}javaw.exe" -jar {}"\%1" \%{*}}
\end{center}
(note that there is a space on either side of -jar).
\item Finally, close all the dialogue windows.  Now all you should have to do is to double-click on the file icon to launch \xlogo!
\end{enumerate}

If that still doesn't work, there is a second possibility.  Open an MSDOS box (on XP: Start -> All Programs -> Accessories -> Command Prompt), and then type in the following command:
\begin{center}
\begin{verbatim}
java -jar \path\to\XLogo

For example : java -jar c:\windows\office\xlogo.jar

\end{verbatim}
(if \texttt{xlogo.jar} is located in this folder).

\end{center}

If you find it annoying to have to keep typing this command, type it into a text file and save it as (say) xlogo.bat.  You can then just double-click on xlogo.bat to launch \xlogo.

\subsubsection*{Associating files with the extension .lgo with XLogo}
Files with the extension .lgo will not usually be recognised by your computer, and when you double-click on them, a dialogue box will appear asking you which application should be used to open files with the .lgo extension. Select \texttt{other} and then give the path to \texttt{javaw.exe} \begin{center}
Usually, this will be: \texttt{C:\textbackslash{}Program Files\textbackslash{}java\textbackslash{}j2re1.4.1\textbackslash{}bin\textbackslash{}javaw.exe}
\end{center}  
You will have to input a name to designate files with an \texttt{.lgo} extension.\\
For example: \texttt{Logo Files}\\
To set this up as a default on Windows XP, follow the steps below:

\begin{enumerate}
\item Start -> Control Panel -> Switch to Classic View  -> Folder options
\item Click on the tab File Types (the third tab)
\item Find in the list of registered file types those connected with jar files (jar files, executable jar files, jar archive, etc)
\item Click the file type, and then click New
\item Type the extension .lgo into the File Extension box, and click OK
\item Click on the newly-added LGO entry in the list of Registered file types, and then click Advanced...
\item A new window will appear: click on New...
\item Under Action, enter "open", and then click on Browse... to navigate to javaw.exe; this is usually
\begin{center}
\texttt{c:\textbackslash{}Program Files\textbackslash{}java\textbackslash{}j2re1.4.1\textbackslash{}bin\textbackslash{}javaw.exe}
\end{center}
\item Click on Open to add the path to the Actions box of the Edit File Type dialogue.
\item Click on open, and then Edit...
\item The path {}``c:\textbackslash{}Program Files\textbackslash{}java\textbackslash{}j2re1.4.1\textbackslash{}bin\textbackslash{}javaw.exe'' will be in the field \textit{Application used to perform action:}.  You need to make an addition to the end of this, so that it reads:
\begin{center}
\texttt{ "c:\textbackslash{}Program Files\textbackslash{}java\textbackslash{}j2re1.4.1\textbackslash{}bin\textbackslash{}javaw.exe" -jar xlogo.jar "\%1" \%{*}}
\end{center}
\item Finally, close all the dialogue windows.  Now all you should have to do is to double-click on the file icon to launch \xlogo!
\end{enumerate}

\section{\xlogo\ Updates}
\begin{center}
\includegraphics{pics/rss.png} \hspace{1cm} \texttt{http://xlogo.tuxfamily.org/rss.xml}
\end{center}
To update \xlogo, you just have to replace the file \texttt{xlogo.jar} with its new version. 
If you want to receive an alert whenever a new version is published, you can subscribe to \xlogo's RSS feed. Its address is 
\begin{center}
 \texttt{http://xlogo.tuxfamily.org/rss.xml}
\end{center}
Several softwares can manage RSS feeds, if you're not familiar with this technique, the easiest way is to use Mozilla Thunderbird:
\begin{itemize}
 \item Menu Edit - Account Settings...
 \item Button ``Add Account''
 \item ``RSS News \& Blogs'', Next
 \item Account Name: ``RSS Feeds'' for example
 \item Buttons ``Next'' and ``Finish''
 \item In the main window ``Account Settings'', Select ``RSS Feeds'' on the left menu and click on button ``Manage Subscriptions''.
 \item Button ``Add''
	\begin{itemize}
 	\item Feed URL: \texttt{http://xlogo.tuxfamily.org/rss.xml}
	\item  Check item  ``Show the article summary instead of loading the web page''
	\end{itemize}
\end{itemize}
\vspace*{0.2cm}
It's done, with the button ``Get Mail'', you can receive \xlogo\ News in the same way you receive mails.
\section{Uninstall}\label{fichier_perso}
To uninstall \xlogo, all that needs to be done is to delete the file \texttt{xlogo.jar} and the configuration file \texttt{.xlogo}\label{file_perso}, which is located in your home directory (\texttt{/home/votre\_login} for Linux users, or \texttt{c:\textbackslash windows\textbackslash.xlogo} for Windows users).
