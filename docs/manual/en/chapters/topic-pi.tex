\chapter{Topic: Probabilistic approximation of $\pi$}
{ }\hfill\textbf{Level:} Advanced\\ \\
\noindent \textsc{Note:} Some elementary mathematical knowledge needed for this chapter.
\section{GCD (Greatest Common Divisor)}
\noindent 
The GCD of two integers is the largest positive integer that divides both numbers without remainder.
\begin{itemize}
\item For example, the GCD of 42 and 28 is 14. (It's the largest integer that divide both 42 and 28)
\item The GCD of 25 and 55 is 5.
\item The GCD of 42 and 23 is 1.
\end{itemize}
The integers $a$ and $b$ are said to be \textbf{coprime} or \textbf{relatively prime} if they have no common factor other than 1 or, equivalently, if their greatest common divisor is 1.
With the precedent example, 42 and 23 are relatively prime.
\section{Euclidean algorithm}
\noindent Calculate the GCD of two inetgers efficiently can be done with the Euclidean algorithm. (Here, we don't show that this algorithm is valid)\\ \\
\textbf{Description of the algorithm:}\\
Given two positive integers $a$ and $b$, we check first if $b$ is equal to 0. If it's the case, then the GCD is equal to $a$. Otherwise, we calculate $r$, the remainder after dividing $a$ by $b$. Then, we replace $a$ by $b$, and $b$ by $r$, and we restart this method. \\
For example, let's calculate the GCD of 2160 and 888  with the Euclidean algorithm:
\begin{center}
$\begin{array}{ccc}
a & b & r\\
2160 & 888 & 384 \\
888 & 384 & 120 \\
384 & 120 & 24 \\
120 & 24 & 0 \\
24 & 0 & \\
\end{array}$
\end{center}
Hence, the GCD of 2160 and 888 is 24. There's no largest integer that divide both numbers.
(In fact $2160=24\times90$ and $888=24\times37$)\\
GCD is the last remainder not equal to 0.
\section{Calculate a GCD in \logo\ programming}
\noindent A little recursive procedure will calculate the GCD of two integers \texttt{:a} and \texttt{:b}
\begin{verbatim}
to gcd :a :b
if (modulo :a :b)=0 [output :b][output gcd :b modulo :a :b] 
end

print gcd 2160 888 
24
\end{verbatim}
Note: It's important to put parenthesis around \texttt{modulo :a :b}. Otherwise, the interpreter would try to evaluate \texttt{:b = 0}. If you don't want to use parenthesis, write: \texttt{if 0=remainder :a :b}

\section{Calculating $\pi$-approximation}
\noindent In fact, a famous result in numbers theory says that the probability for two randomly chosen integers to be coprime is  $\dfrac{6}{\pi^2}\approx 0,6079$. To exhibit this result, we're going to:
\begin{itemize}
\item Choose randomly two integers between 0 and 1 000 000.
\item Calculate their GCD.
\item If the GCD value is 1, increment the counter variable.
\item Repeat this experience 1000 times.
\item The frequence for the  couple of coprime integers can be calculated dividing the variable counter by 1000 (tries number).
\end{itemize}
\begin{verbatim}

to test
# We set the variable counter to 0
globalmake "counter 0
repeat 1000 [ 
  if (gcd random 1000000 random 1000000)=1 [globalmake "counter :counter+1]
]
print [frequence:]
print :counter/1000
end

\end{verbatim}
In a similar way as the precevious note, notice the parenthesis around \texttt{gcd random 1000000 random 1000000}. Otherwise, the interpreter will try to evaluate $1\ 000\ 000 = 1$. You can write in other way: \texttt{if 1=gcd random 1000000 random 1000000}\\ \\
We execute the program \texttt{test}.
\begin{verbatim}
test
0.609
test
0.626
test
0.597
\end{verbatim}
We obtain some values close to the theorical probability: 0,6097. This frequency is an approximation of $\dfrac{6}{\pi^2}$.\\
 Thus, if I note $f$ the frequency, we have: $f\approx \dfrac{6}{\pi^2}$ \\
Hence, $\pi^2\approx\dfrac{6}{f}$ and $\pi\approx\sqrt{\dfrac{6}{f}}$.\\
I append to my program a line that gives this $\pi$ approximation in procedure \texttt{test}:
\begin{verbatim}
to test
# We set the variable counter to 0
globalmake "counter 0
repeat 1000 [ 
  if (gcd random 1000000 random 1000000)=1 [globalmake "counter :counter+1]
]
# We calculate te frequency
make "f :counter/1000
# we dispaly the pi approximation
print sentence  [ pi approximation:] sqrt (6/:f)
end

test
pi approximation: 3.1033560252704917 
test
pi approximation: 3.1835726998350666 
test
pi approximation: 3.146583877637763 

\end{verbatim}
 Now, we modify the program because I want to set the number of tries. I want to try with 10\ 000 and perhaps more tries.
\begin{verbatim}
to test :tries
# We set the variable counter to 0
globalmake "counter 0
repeat :tries [ 
  if (gcd random 1000000 random 1000000)=1 [globalmake "counter :counter+1]
]
# We calculate te frequency
make "f :counter/:tries
# we dispaly the pi approximation
print sentence  [ pi approximation:] sqrt (6/:f)
end

test 10000
pi approximation: 3.1426968052735447 
test 10000
pi approximation: 3.1478827771265787 
test 10000
pi approximation: 3.146583877637763 
test 10000

\end{verbatim} 
 Quite interesting, isn't it?
\section{More complex: $\pi$ generating $\pi$.....}
What is a random integer? Is an integer choosen randomly between 1 and 1000000 really representative for all integers choosen randomly? We can see that our experience is only an approximation of an ideal model. Here, we're going to modify the method for generating random integers... We won't use the primitive \texttt{random}, we're going to generate random integers with the $\pi$ digits sequence.\\
$\pi$ digits have always interested mathematicians:
\begin{itemize}
 \item The numbers 0 to 9, do some appear more often than others?
 \item Is there some sequence of integers that appear frequently?
\end{itemize}
In reality, it \textbf{seems} that the $\pi$ digit sequence is a really randomly sequence. (Result not demonstrated yet). It's not possible to predict the following digit after the others, there's no period.\\ \\
Here is the method we're going to use to generate integers randomly choosen:
\begin{itemize}
 \item First, we need the first digit of $\pi$ (For example, one billion) 
	 \begin{enumerate}
 		\item First way: some programs calculate the $\pi$ digits. For example, PiFast in Windows environment and SchnellPi for Linux.
 		\item Second way: you can download this file from \xlogo\ website: 
		\begin{center}
		\texttt{http://downloads.tuxfamily.org/xlogo/common/millionpi.txt} 
		\end{center}
 	\end{enumerate}
\item To generate the integers, we're going to read the digits sequence in packet of 7 digits:\\
$\underbrace{3.1415926}_{\textrm{First number}}\underbrace{53589793}_{\textrm{Second number}}\underbrace{23846264}_{\textrm{Third number}}338327950288419716939$ etc\\ \\
I remove the point ``. '' in  3.14 ....that will cause problem when we're going to extract the digits
\end{itemize}
Let's create now a new procedure called \texttt{randompi} and let's modify the procedure \texttt{test}
\begin{verbatim}
to gcd :a :b
if (modulo :a :b)=0 [output :b][output gcd :b modulo :a :b] 
end

to test :tries
# We open a flow whose identifier is 1 towards the file  millionpi.txt
# Here we suppose that millionpi.txt is in the current directory
# Otherwise, fix it with changedirectory
openflow 1 "millionpi.txt
# Set the variable line to the first line of the file millionpi.text
globalmake "line first readlineflow 1
# Set the variable counter to  0
globalmake "counter 0
repeat :tries [
  if 1=gcd randompi 7 randompi 7 [globalmake "counter :counter+1]
]
# Calculate frequency
globalmake "f :counter/:tries
# Display th pi approximation
print sentence [ pi approximation:] squareroot (6/:f)
closeflow 1
end

to randompi :n
localmake "number "
repeat :n [
# If there's no char yet on the line
if 0=count :line [globalmake "line first readlineflow 1]
# Set the variable char to the first character of the line
globalmake "char first :line
# Then remove  first character from the line.
globalmake "line butfirst :line
globalmake "number word :number :char
]
output :number
end

test 10
approximation de pi: 3.4641016151377544 
test 100
approximation de pi: 3.1108550841912757 
test 1000
approximation de pi: 3.081180112566604 
test 10000
approximation de pi: 3.1403714651066386 
test 70000
approximation de pi: 3.1361767950325627
\end{verbatim}
We find a correct approximation of $\pi$ with its own digits!\\ \\
It's still possible to imrove the program by indicating the time for the computation. We add on the first line of the procedure \texttt{test}:
\begin{center}
\texttt{globalmake "begin pasttime}
\end{center}
Then we append before \texttt{closeflow}:
\begin{center}
\texttt{print sentence [pasttime mis: ] pasttime - :begin}
\end{center}
