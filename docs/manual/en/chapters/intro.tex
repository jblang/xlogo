\chapter{Introduction}
\logo\ is a programming language developed in the 1960's by Seymour Papert. Papert was the developer of an original and highly influential theory on learning called ``constructionism'' which could be summarised with the expression: ``learning by doing''.\\ \\
\logo\ is a really good language to develop mathematical and logic skills. It is an excellent language to begin learning with, as it teaches the basics of things like loops, tests, procedures, etc. The user moves an object called a "turtle" around the screen using commands as simple as forward, back, right, and so on. As it moves, the turtle leaves a trail behind it, and so it is therefore possible to create drawings. The fact that the user can give the turtle orders in a very natural language makes \logo\ very easy to learn. More advanced usage is possible too with operations on lists, words or files.\\ \\
\xlogo\ is a \logo\ interpreter, it means that the user's instructions are executed directly. The user can see their errors on screen immedately. This very intuitive graphical approach makes Logo an ideal language for beginners, especially children!\\ \\
The main adress for the \xlogo website is
\begin{center}
\texttt{http://xlogo.tuxfamily.org/}
\end{center}
Here you can download both the documentation and the software. You can also find many examples created with \xlogo\ and you will be able to judge \xlogo's capacity.\\ \\
\xlogo\ now supports ten languages (arabic, asturian, english, esperanto, french, galician, greek, german, portuguese and spanish) and is written in \textsc{Java} - a programming language with the benefit of being cross-platform. Therefore \xlogo\ will run on Linux, Windows or MacOS machines without problems.\\ \\
\xlogo\  is licensed under the GPL: Hence, it is free software and users have four freedoms:
\begin{itemize}
\item Freedom 1: The freedom to run the program for any purpose.
\item Freedom 2: The freedom to study and modify the program.
\item Freedom 3: The freedom to copy the program so you can help your neighbour.
\item Freedom 4: The freedom to improve the program, and release your improvements to the public, so that the whole community benefits.
\end{itemize}
\vspace{0.3cm}
\noindent \textbf{Manual structure:}\\ \\
This manual will help you to discover \xlogo.
\begin{itemize}
 \item In the first part, different menus and interface options are explained.
 \item Then, some chapters presenting the most important instructions to begin using \xlogo. The first are very easy and then, complexity grows. Sometimes, at the end of a chapter, some exercices are presented. Their solutions can be found in appendix D.
\item Then, a sequence of different themes is offered for advanced users.
\item In appendix A, you'll find a complete description of all \xlogo's primitives.
\end{itemize}
\vspace{0.5cm}
This manual exists under several formats:
\begin{itemize}
 \item \textsc{PDF}: http://downloads.tuxfamily.org/xlogo/downloads-en/manual-en.pdf
 \item \textsc{Zipped HTML}: http://downloads.tuxfamily.org/xlogo/downloads-en/manual-html-en.zip
 \item \LaTeXe: Manual Source: http://downloads.tuxfamily.org/xlogo/downloads-fr/manual-src-en.zip
 \item \textsc{JavaHelp}: Menu Help-Online Manual in \xlogo
\end{itemize}
