\chapter{Utilizaci\'on de la red con \textsc{XLogo}}
  \label{Red_XLogo}

\section{La red: ?`c\'omo funciona eso?}
  \label{Red_Como}
En primer lugar es necesario explicar los conceptos b\'asicos de la
comunicaci\'on en una red para usar correctamente las primitivas de
\textsc{XLogo}. \\
\begin{center}
\includegraphics[scale=0.5]{Imagenes/12_Redes/Redes_es.png}

Figura: noci\'on de red
\end{center}

Dos ordenadores pueden comunicarse a trav\'es de una red si est\'an equipados
con una tarjeta de red (llamada tambi\'en tarjeta \textit{ethernet}). Cada
ordenador se  identifica por una direcci\'on personal: su direcci\'on I.P.
Esta direcci\'on IP consta de cuatro n\'umeros enteros comprendidos entre
0 y 255 separados por puntos. Por ejemplo, la direcci\'on IP del primer
ordenador del esquema de la figura es \texttt{192.168.1.1} \\

Dado que no es f\'acil acordarse de este tipo de direcci\'on, tambi\'en se
puede hacer corresponder a cada direcci\'on IP un nombre m\'as f\'acil de
recordar. Sobre el esquema anterior, podemos comunicar con el ordenador de
la derecha bien llam\'andolo por su direcci\'on IP: \texttt{192.168.1.2}, o
llam\'andolo por su nombre: \texttt{tortuga}. \\

No nos extendamos m\'as sobre el significado de estos n\'umeros. A\~nadamos
\'unicamente una cosa que es interesante saber, el ordenador local en el cual
se trabaja tambi\'en se identifica por una direcci\'on: \texttt{127.0.0.1}.
El nombre que se asocia con \'el es habitualmente \texttt{localhost}.

\section{Primitivas orientadas a la red}
  \label{Red_Primitivas}
\textsc{XLogo} dispone de 4 primitivas que permiten comunicarse a trav\'es de
una red: \texttt{escuchatcp}, \texttt{ejecutatcp}, \texttt{chattcp} y
\texttt{enviatcp}. 

En los ejemplos siguientes, mantendremos el esquema de red de la subsecci\'on
anterior.
\begin{itemize}
   \item \texttt{escuchatcp}: \index{escuchatcp@\texttt{escuchatcp}} esta
      primitiva es la base de cualquier comunicaci\'on a trav\'es de la red.
      No espera ning\'un argumento. Permite poner al ordenador que la ejecuta
      a la espera de instrucciones dadas por otros ordenadores de la red.
   \item \texttt{ejecutatcp}: \index{ejecutatcp@\texttt{ejecutatcp}} esta
      primitiva permite ejecutar instrucciones sobre otro ordenador de la red.

      Sintaxis: \texttt{ejecutatcp palabra lista} $\rightarrow$ La palabra indica
      la direcci\'on IP o el nombre del ordenador de destino (el que va a
      ejecutar las \'ordenes), la lista contiene las instrucciones que hay
      que ejecutar. \\

      Ejemplo: desde el ordenador \texttt{liebre}, deseo trazar un cuadrado
      de lado 100 en el otro ordenador. Por tanto, hace falta que desde el
      ordenador \texttt{tortuga} ejecute la orden \texttt{escuchatcp}. Luego,
      desde el ordenador \texttt{liebre}, ejecuto:
      \begin{verbatim}
 ejecutatcp "192.168.2.2 [repite 4 [avanza 100 giraderecha 90]] \end{verbatim}
      o
      \begin{verbatim}
 ejecutatcp "tortuga [repite 4 [avanza 100 giraderecha 90]] \end{verbatim}

   \item \texttt{chattcp}: \index{chattcp@\texttt{chattcp}} permite chatear
      entre dos ordenadores de la red, abriendo una ventana en cada uno que
      permite la conversaci\'on.

      Sintaxis: \texttt{chattcp palabra lista} $\rightarrow$ La palabra indica la
      direcci\'on IP o el nombre del ordenador de destino, la lista contiene
      la frase que hay que mostrar.

      Ejemplo: \texttt{liebre} quiere hablar con \texttt{tortuga}. \\
      \texttt{tortuga} ejecuta \texttt{escuchatcp} para ponerse en espera de
      los ordenadores de la red. \texttt{liebre} ejecuta entonces:
      \texttt{chattcp "192.168.1.2 [buenos d\'ias]}.\\
      Una ventana se abre en cada uno de los ordenadores para permitir la
      conversaci\'on 
   \item \texttt{enviatcp}: \index{enviatcp@\texttt{enviatcp}} env\'ia datos
      hacia un ordenador de la red.

      Sintaxis: \texttt{enviatcp palabra lista} $\rightarrow$ La palabra indica
      la direcci\'on IP o el nombre del ordenador de destino, la lista
      contiene los datos que hay que enviar. Cuando \textsc{XLogo} recibe
      los datos en el otro ordenador, responder\'a \texttt{Si} , que podr\'a
      asignarse a una variable o mostrarse en el \textbf{Hist\'orico de
      comandos}.\\ 

      Ejemplo: \texttt{tortuga} quiere enviar a \texttt{liebre} la frase
      \texttt{\char`\"{}3.14159 casi el n\'umero pi"}. \\
      \texttt{liebre} ejecuta \texttt{escuchatcp} para ponerse en espera de
      los ordenadores de la red. \\
      Si \texttt{tortuga} ejecuta entonces:
      \texttt{enviatcp "liebre [3.14159 casi el n\'umero pi]}, \texttt{liebre}
      mostrar\'a la frase, pero en \texttt{tortuga} aparecer\'a el mensaje:\\
      \textcolor{red}{\texttt{?`Qu\'e hacer con  [ Si ]  ?}}\\
      Deber\'iamos escribir: \\
      \texttt{es enviatcp "liebre [3.14159 casi el n\'umero pi]} \\
      o \\
      \texttt{haz \char`\"{}respuesta enviatcp \char`\"{}liebre [3.14159 casi el 
            n\'umero pi]}\\
      En el primer caso, el \textbf{Hist\'orico de comandos} mostrar\'a
      \texttt{Si}, y en el segundo \verb+"respuesta+ contendr\'a la lista 
      \texttt{[ Si ]}, como podemos comprobar haciendo\\
      \texttt{es lista? :respuesta} \\
      \texttt{cierto} \\
      \texttt{es :respuesta} \\
      \texttt{Si} \\

      Con esta primitiva se puede establecer comunicaci\'on con un robot
      did\'actico a trav\'es de su interfaz de red. En este caso, la respuesta
      del robot puede ser diferente, y se podr\'an decidir otras acciones en
      base al contenido de \texttt{:respuesta}.
\end{itemize}

\noindent Un peque\~no truco: lanzar dos veces \textsc{XLogo} en un mismo
ordenador.
\begin{itemize}
   \item En la primera ventana, ejecuta \texttt{escuchatcp}.
   \item En la segunda, ejecuta
      \begin{verbatim}
 ejecutatcp "127.0.0.1 [repite 4 [avanza 100 giraderecha 90]] \end{verbatim}
\end{itemize}
!`As\'i puedes mover a la tortuga en la otra ventana! (!`Ah s\'i!, esto es
as\'i porque \texttt{127.0.0.1} indica tu direcci\'on local, es decir, de
tu propio ordenador)
