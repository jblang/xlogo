\chapter{Modo multitortuga}
   \label{Modo-multitortuga}
   \index{multitortuga}

Se pueden tener varias tortugas activas en pantalla. Nada m\'as iniciarse
\textsc{XLogo}, s\'olo hay una tortuga disponible. Su n\'umero es 0.
Si quieres ``crear'' una nueva tortuga, puedes usar la primitiva
\texttt{pontortuga} \index{pontortuga@\texttt{pontortuga}}
seguida del n\'umero de la nueva tortuga. Para evitar confusi\'on, la
nueva tortuga se crea en el centro y es invisible (tienes que usar
\texttt{muestratortuga} \index{muestratortuga@\texttt{muestratortuga}}
para verla). As\'i, la nueva tortuga es la activa, y ser\'a la que obedezca
las cl\'asicas primitivas mientras no cambies a otra tortuga con
\texttt{pontortuga}. El m\'aximo n\'umero de tortugas disponibles puede
fijarse en el men\'u \textbf{Herramientas $\rightarrow$ Preferencias}. \\

Estas son las primitivas que se aplican al modo multitortuga:
\begin{center} \begin{longtable}{|m{3cm}|c|m{9cm}|} \hline 
   \multicolumn{1}{|c|}{\textbf{Primitiva}} &
      \multicolumn{1}{c|}{\textbf{Argumentos}} &
         \multicolumn{1}{c|}{\textbf{Uso}} \\ \endhead \hline 
   \texttt{pontortuga}, \index{pontortuga@\texttt{pontortuga}}
      \texttt{ptortuga}\index{ptortuga@\texttt{ptortuga}} & 
         \texttt{a: n\'umero} &
        La tortuga n\'umero \texttt{a} es ahora la tortuga activa. Por defecto,
        cuando \textsc{XLogo} comienza, est\'a activa la tortuga n\'umero
        0.\\ \hline 
   \texttt{tortuga} \index{tortuga@\texttt{tortuga}} & \texttt{no} &
        Da el n\'umero de la tortuga activa. \\ \hline 
   \texttt{tortugas} \index{tortugas@\texttt{tortugas}} & \texttt{no} &
        Da una lista que contiene todos los n\'umeros de tortuga actualmente
        en pantalla. \\ \hline 
   \texttt{eliminatortuga} \index{eliminatortuga@\texttt{eliminatortuga}} &
       \texttt{a: n\'umero} &
        Elimina la tortuga n\'umero \texttt{a} \\ \hline
   \texttt{ponmaximastortugas},
       \index{ponmaximastortugas@\texttt{ponmaximastortugas}}
   \texttt{pmt} \index{pmt@\texttt{pmt}} &\texttt{n: n\'umero} &
        Fija el m\'aximo n\'umero de tortugas \\ \hline 
   \texttt{maximastortugas}, \index{maximastortugas@\texttt{maximastortugas}}
   \texttt{maxt} \index{maxt@\texttt{maxt}} & \texttt{no} &
        Devuelve el m\'aximo n\'umero de tortugas \\ \hline 
\end{longtable} \end{center}