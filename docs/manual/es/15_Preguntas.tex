\chapter{Carnaval de Preguntas -- Artima\~nas -- Trucos que conocer}
   \label{PreguntasFrecuentes}

\section{Preguntas frecuentes}

\subsection*{Por m\'as que borro un procedimiento en el Editor, reaparece
            todo el tiempo!}
   \label{Borrar-Editor}

Cuando se sale del \textbf{Editor}, \'este se limita \'unicamente a guardar
o poner al d\'ia los procedimientos definidos en \'el. La \'unica forma
de borrar un procedimiento en \textsc{XLogo} es utilizar la primitiva
\texttt{borra} o \texttt{bo}.

\textbf{Ejemplo}: \verb+borra "toto+ $\rightarrow$ borra
el procedimiento \texttt{toto}


\subsection*{?`C\'omo hago para escribir r\'apidamente una orden utilizada
            previamente?}
    \label{Recuperar-Ordenes}

\begin{itemize}
   \item Primer m\'etodo: con el rat\'on, haz \textit{click} en la zona del
      \textbf{Hist\'orico} sobre la l\'inea deseada. As\'i reaparecer\'a
      inmediatamente en la \textbf{L\'inea de Comando}
   \item Segundo m\'etodo: con el teclado, las flechas Arriba y Abajo
      permiten navegar por la lista de los comandos anteriores (m\'as
      pr\'actico)
\end{itemize}

\subsection*{En la opci\'on Sonido del cuadro de di\'alogo Preferencias, 
            no hay disponible ning\'un instrumento}
   \label{Sonido-Instrumentos-Error}

Como dec\'iamos en la secci\'on \ref{Menu-Herramientas}, esto se debe
a que la versi\'on de \textsc{Java} para Windows no incluye los 
\textit{bancos de sonido}, y deben instalarse manualmente.

En primer lugar, hay que descargarlos desde:

\noindent
\texttt{http://java.sun.com/products/java-media/sound/soundbank-min.gm.zip} \\
la versi\'on m\'inima (unos 350 kb),

\noindent
\texttt{http://java.sun.com/products/java-media/sound/soundbank-mid.gm.zip} \\
la versi\'on intermedia (algo m\'as de 1 Mb) y

\noindent \texttt{http://java.sun.com/products/java-media/sound/soundbank-deluxe.gm.zip} \\
la versi\'on \textit{de luxe} (casi 5 Mb). \\

Una vez descargados, debemos descomprimirlos en el directorio \texttt{audio} de
la instalaci\'on \textsc{Java} que, dependiendo de la versi\'on, puede ser:
\begin{verbatim}
 C:\Archivos de programa\Java\jre1.6.0\lib\audio\end{verbatim}
creando el directorio \texttt{audio} si \'este no existe. \\

\noindent Hecho esto, la lista de instrumentos estar\'a disponible.

\subsection*{Tengo problemas de refresco de pantalla cuando la tortuga dibuja!}
   \label{Problemas-Refresco}

Problema tambi\'en conocido y t\'ipico de JRE > 1.4.2. intentar\'e ponerle
remedio en lo sucesivo, quiz\'a pueda hacer algo. Dos soluciones por
el momento:
\begin{itemize}
   \item Minimizar la ventana y volver a aumentarla (restaurarla)
   \item Utilizar siempre la versi\'on m\'as moderna de \textsc{Java}.
\end{itemize}

\subsection*{Utilizo la versi\'on en Esperanto, pero no puedo escribir los
            caracteres especiales}
   \label{Caracteres-Especiales-Esperanto}

Cuando escribes en la \textbf{L\'inea de comandos} o en el \textbf{Editor},
si haces \textit{click} con el bot\'on derecho del rat\'on, aparece un men\'u
contextual. En ese men\'u se encuentran las funciones habituales de Edici\'on
(copiar, cortar, pegar) y los caracteres especiales del Esperanto cuando se
selecciona ese idioma.

\subsection*{Utilizo la versi\'on en Espa\~nol, y no puedo utilizar las
            primitivas \texttt{animacion}, \texttt{division}, 
            \texttt{separacion} y \texttt{ponseparacion}
    \label{Primitivas-Espanol-Errores}}

Corregido desde la versi\'on \texttt{0.9.20e}. Para versiones anteriores
de \textsc{XLogo}:
\begin{itemize}
   \item \texttt{animacion}, se escribe \texttt{animacicn}, con
      ``\texttt{c}'' en lugar de ``\texttt{o}''. 
   \item \texttt{division}, \texttt{separacion} y \texttt{ponseparacion} se
      escriben con tilde: \texttt{divisi\'on}, \texttt{separaci\'on} y
      \texttt{ponseparaci\'on}.
\end{itemize}
Obviamente, el mejor consejo es que actualices a la versi\'on m\'as moderna
de \textsc{XLogo}.

\subsection*{Uso Windows XP y tengo correctamente instalado y configurado el
    JRE; pero hago doble \textit{click} en el icono de \textsc{XLogo} y no pasa
    nada!!}
   \label{Problema-WinXP}

A veces en la primera ejecuci\'on de \textsc{XLogo} en Windows XP pasa eso. Dos
opciones:
\begin{itemize}
   \item Utiliza la versi\'on \texttt{xlogo-new.jar} tambi\'en disponible
      en nuestra web.
   \item Si presionas \texttt{Alt+Contrl+Supr} y en el \textbf{Gestor de
      Procesos} ``matas'' el correspondiente a \texttt{javaw},
      se inicia \textsc{XLogo}. Desde ese momento, funciona correctamente
      haciendo ``doble \textit{click}'' sobre el icono del
      archivo \texttt{xlogo.jar}. 
\end{itemize}

\section{?`C\'omo puedo ayudar?}
   \label{Como-ayudar?}

\begin{itemize}
   \item Inform\'andome de todos los errores (``\textit{bugs}'')
      que encuentres. Si puedes reproducir sistem\'aticamente un problema
      detectado, mejor a\'un
   \item Toda sugerencia dirigida a mejorar el programa es siempre bienvenida
   \item Ayudando con las traducciones, en particular el ingl\'es \ldots{}
   \item Las palabras de \'animo siempre vienen bien
\end{itemize}
