\chapter{Ejecutando \textsc{XLogo} desde la \textit{web}}

\section{El problema}

Supongamos que usted es administrador/a de un sitio web. En este sitio, habla de
\textsc{XLogo} y quiere ofrecer algunos de los programas que ha creado.

Podr\'ia distribuir el fichero \textsc{Logo} en formato \texttt{.lgo} y esperar
que los visitantes descarguen primero \textsc{XLogo}, despu\'es el programa y, por
\'ultimo, lo carguen para ver qu\'e hace. \\

\noindent La alternativa es que el usuario pueda ejecutar \textsc{Xlogo} 
\textit{en l\'inea}, y probar su programa sin apenas esfuerzo. \\

\noindent Para lanzar \textsc{XLogo} desde una p\'agina \textit{web}, utilizaremos
la tecnolog\'ia \textsc{Java Web Start}. Con ello, s\'olo necesita poner en su
web un enlace hacia un archivo con extensi\'on \texttt{.jnlp}, que iniciar\'a la
ejecuci\'on de \textsc{XLogo}.

\section{C\'omo crear un fichero \texttt{.jnlp}}

Veamos c\'omo hacerlo con un archivo de ejemplo. De hecho, el siguiente ejemplo es
el que se usa en la secci\'on ``\textbf{Ejemplos}'' de la p\'agina de \textsc{XLogo} 
en franc\'es. \\

\noindent Este archivo carga el programa que dibuja un dado en la secci\'on 3D.
Veamos primero el archivo, y veremos despu\'es las explicaciones sobre su contenido.

\begin{verbatim}
<?xml version="1.0" encoding="utf-8"?>
<jnlp spec="1.5+" codebase="http://downloads.tuxfamily.org/xlogo/common/webstart">
<information>
  <title>XLogo</title>
  <vendor>xlogo.tuxfamily.org</vendor>
  <homepage href="http://xlogo.tuxfamily.org"/>
  <description>Logo Programming Language</description>
  <offline-allowed/>
</information>

<security>
  <all-permissions/>
</security>

<resources>
  <j2se version="1.4+"/>
  <jar href="xlogo.jar"/>
</resources>

<application-desc main-class="Lanceur">
  <argument>-lang</argument>
  <argument>fr</argument>
  <argument>-a</argument>
  <argument>http://xlogo.tuxfamily.org/fr/html/examples-fr/3d/de.lgo</argument>
</application-desc>
</jnlp>
\end{verbatim}

Este archivo est\'a escrito en formato XML, y las cuatro l\'ineas m\'as importantes
son:
\begin{verbatim}
  <argument>-lang</argument>
  <argument>fr</argument>
  <argument>-a</argument>
  <argument>http://xlogo.tuxfamily.org/fr/html/examples-fr/3d/de.lgo</argument>
\end{verbatim}
Esas l\'ineas especifican los par\'ametros de inicio de \textsc{XLogo}:
\begin{itemize}
   \item Las l\'ineas 1 y 2 fuerzan a \textsc{XLogo} a ejecutarse en franc\'es.
   \item La \'ultima línea indica la ruta del fichero a cargar.
   \item La l\'inea 3 indica que se ejecute el Comando de Inicio de este
      programa al iniciarse \textsc{XLogo}. 
\end{itemize}

Una \'ultima sugerencia (petici\'on): Dado que el servidor de Tuxfamily no puede
aceptar ``infinitas'' conexiones, le pedimos que tenga una copia del archivo
\texttt{xlogo.jar} en su web. Para ello, cambie la direcci\'on en la segunda
l\'inea del archivo \texttt{.jnlp}, donde dice \texttt{codebase=} por:
\begin{verbatim}
   <jnlp spec="1.5+" codebase="http://mi.direccion.web/directorio/XLogo">
\end{verbatim}
