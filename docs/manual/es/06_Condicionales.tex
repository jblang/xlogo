\chapter{Condicionales}
   \label{Condicionales}
   \index{Condicionales}

La primitiva b\'asica que define el condicional en \textsc{XLogo} es \texttt{si}.
\index{si@\texttt{si}}Su uso es simple: \\

\texttt{si expresi\'on\_l\'ogica [comandos]} \\

\noindent que ejecuta \texttt{comandos} \'unicamente cuando
\texttt{expresi\'on\_l\'ogica} sea \texttt{cierto}, o bien: \\

\texttt{si expresi\'on\_l\'ogica [comandos1] [comandos2]} \\

\noindent donde \texttt{comandos1} y \texttt{comandos2} son, respectivamente,
las \'ordenes a ejecutar en los casos en los que \texttt{expresi\'on\_l\'ogica}
sea \texttt{cierto} o \texttt{falso}.

\begin{quote}
\noindent \textbf{Ejemplos: }
\begin{itemize}
   \item Procedimiento que compara un n\'umero dado con 4 y contesta
      \texttt{MAYOR} si el n\'umero es mayor que 4: 
      \begin{verbatim}
   para mayor :X
     si :x > 4 [escribe "MAYOR]
   fin \end{verbatim}
   \item Procedimiento que compara un n\'umero con 4, para ver si es
      mayor que 4 o no lo es: 
      \begin{verbatim}
   para compara :X
     si :x > 4 [escribe "SI] [escribe "NO]
   fin \end{verbatim}
\end{itemize}
\end{quote}

Si queremos que los argumentos para \texttt{cierto} y \texttt{falso} est\'en
guardados en sendas variables, no podemos usar \texttt{si}. En este caso, la
primitiva correcta es:
\begin{center}
   \texttt{sisino}\index{sisino@\texttt{sisino}}
\end{center}
En este ejemplo, \textsc{XLogo} mostrar\'a un error:
\begin{verbatim}
  haz "Opcion_1 [escribe "cierto]
  haz "Opcion_2 [escribe "falso]
  si 1 = 2 :a :b\end{verbatim}
ya que la segunda lista nunca ser\'a evaluada:

\noindent\textcolor{red}{\texttt{?`Qu\'e hacer con [escribe \char`\"{}falso]?}}

La sintaxis correcta es:
\begin{verbatim}
  haz "Opcion_1 [escribe "cierto]
  haz "Opcion_2 [escribe "falso]
  sisino 1 = 2 :a :b\end{verbatim}
que devolver\'a:
\begin{verbatim}
  "falso\end{verbatim}
