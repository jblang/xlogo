\chapter{Argomento: approssimazione probabilistica di pi greco}

{ }\hfill\textbf{Livello:} Avanzato \\

\textbf{Nota}: Alcune nozioni matematiche elementari sono necessarie per questo capitolo.



\section{MCD (Massimo Comune Divisore)}

Il MCD di due numeri interi è l'intero più grande che divide entrambi i numeri senza resto. Per esempio il MCD di 42 e 28 è 14, il MCD di 25 e 55 è 5, il MCD di 42 e 23 è 1.\\

Gli interi $a$ e $b$ si dicono \textbf{coprimi} o \textbf{primi tra loro} se non hanno un fattore comune eccetto 1 o, in modo equivalente, se il loro MCD è 1. Negli esempi precedenti 42 e 23 sono coprimi.




\section{L'algoritmo di Euclide}
L'algoritmo di Euclide calcola il MCD di due interi in modo efficiente (non dimostriamo qui che questo algoritmo è valido).\\

\subsection{Descrizione dell'algoritmo}
Dati due interi positivi $a$ e $b$, controlliamo prima di tutto se $b$ sia uguale a 0. Se è questo il caso allora il MCD è $a$. In caso contrario calcoliamo $r$ come resto della divisione di $a$ per $b$. Quindi sostituiamo $a$ con $b$ e $b$ con $r$ e ricominciamo l'algoritmo.\\
Per esempio calcoliamo il MCD di 2160 e 888 con l'algoritmo di Euclide:
\begin{center}
	$
	\begin{array}{ccc}
		a & b & r\\
		2160 & 888 & 384 \\
		888 & 384 & 120 \\
		384 & 120 & 24 \\
		120 & 24 & 0 \\
		24 & 0 & \\
	\end{array}
	$
\end{center}
Quindi il MCD di 2160 e 888 è 24. Non c'è intero più grande che divide entrambi i numeri (infatti $2160=24\times90$ e $888=24\times37$)\\
Il MCD è resto più grande non uguale a 0.



\section{Calcolare il MCD in \logo}
Una semplice procedure ricorsiva calcolerà il MCD di due interi \texttt{:a} e \texttt{:b}:
\begin{lstlisting}
Per MCD :a :b
	Se (modulo :a :b)=0 [output :b][output MCD :b modulo :a :b] 
Fine
\end{lstlisting}

Invochiamo la procedure come \texttt{Print MCD 2160 888} e otterremo il risultato 24.\\
Nota: è importante porre tra parentesi tonde \texttt{modulo :a :b} altrimenti l'interprete tenterà di verificare la condizione \texttt{:b = 0}. Se non vuoi usare le parentesi scrivi \texttt{Se 0=resto :a :b}.



\section{Calcolare l'approssimazione di pi greco}
Nei fatti un famoso risultato nella teoria dei numeri dice che la probabilità di due interi scelti a caso di essere coprimi è $\dfrac{6}{\pi^2}\approx 0,6079$. Per verificare questo risultato possiamo:
\begin{itemize}
	\item Scegliere a caso due interi tra 0 e 1000000.
	\item Calcolare il loro MCD.
	\item Se il MCD è 1, incrementiamo una variabile contatore.
	\item Ripetiamo questa procedura per 1000 volte.
	\item La frequenza può essere calcolata dividendo la variabile contatore per 1000 (il numero di prove).
\end{itemize}

\begin{lstlisting}
Per test
	# impostiamo la variabile contatore a 0
	AssegnaVar "counter 0
	Ripeti 1000 [ 
		Se (MCD Casuale 1000000 Casuale 1000000)=1 [AssegnaVar "counter :counter+1]
	]
	Stampa [frequenza:] 
	Stampa :counter/1000
Fine
\end{lstlisting}

Come prima nota le parentesi attorno \texttt{MCD Casuale 1000000 Casuale 1000000}. Se non ci fossero l'interprete cercherebbe di verificare la condizione $1\ 000\ 000 = 1$. La stessa espressione si può anche scrivere come: \texttt{Se 1=MCD Casuale 1000000 Casuale 1000000}. \\

Invochiamo la procedura con \texttt{test} e con un po' di pazienza otterremo frequenze che si avvicinano al valore teorico di 0,6097:\\
\begin{verbatim}
test
0.609
test
0.626
test
0.597
\end{verbatim}

La frequenza teorica è una approssimazione di $\dfrac{6}{\pi^2}$.\\

Quindi, se denotiamo con $f$ la frequenza abbiamo: $f\approx \dfrac{6}{\pi^2}$ da cui $\pi^2\approx\dfrac{6}{f}$ e $\pi\approx\sqrt{\dfrac{6}{f}}$.\\

Aggiungo al mio programma una riga che fornisca questa approssimazione di pi greco nella procedura \texttt{test}:
\begin{lstlisting}
Per test
	# impostiamo la variabile contatore a 0
	AssegnaVar "counter 0
	Ripeti 1000 [ 
	  Se (MCD Casuale 1000000 Casuale 1000000)=1 [AssegnaVar "counter :counter+1]
	]
	# Calcoliamo la frequenza
	Assegna "f :counter/1000
	# stampiamo l'approssimazione di pi greco
	Stampa frase [ approssimazione di pi greco:] RadQ (6/:f)
Fine
\end{lstlisting}
Invochiamo la procedura con \texttt{test} e con un po' di pazienza otterremo valori di pi greco che si avvicinano a quello teorico:

\begin{verbatim}
test
approssimazione di pi greco: 3.1033560252704917 
test
approssimazione di pi greco: 3.1835726998350666 
test
approssimazione di pi greco: 3.146583877637763 
\end{verbatim}

Ora voglio modificare il programma perché vorrei impostare liberamente il numero di prove. Vorrei provare con 10000 e forse con ancor più prove.
\begin{lstlisting}
Per test :prove
	# impostiamo la variabile contatore a 0
	AssegnaVar "counter 0
	Ripeti :prove [ 
	  Se (MCD Casuale 1000000 Casuale 1000000)=1 [AssegnaVar "counter :counter+1]
	]
	# Calcoliamo la frequenza
	Assegna "f :counter/:prove
	# stampiamo l'approssimazione di pi greco
	Stampa frase [ approssimazione di pi greco:] RadQ (6/:f)
Fine
\end{lstlisting}

Invochiamo la procedura così:
\begin{verbatim}
test 10000
approssimazione di pi greco: 3.1426968052735447 
test 10000
approssimazione di pi greco: 3.1478827771265787 
test 10000
approssimazione di pi greco: 3.146583877637763 
\end{verbatim} 

Interessante, no?



\section{La generazione del pi greco mediante il pi greco\textellipsis}
Che cos'è un intero casuale? Può un intero scelto casualmente tra 1 e 1000000 essere realmente rappresentativo di tutti gli interi scelti casualmente? Osserviamo che il nostro esperimento è solo una approssimazione del modello ideale. Adesso modificheremo il metodo per generare numeri casuali\textellipsis. Non useremo la primitiva \texttt{Casuale} ma genereremo numeri casuali usando la sequenza delle cifre del pi greco.\\
Le cifre del pi greco hanno sempre interessato i matematici:
\begin{itemize}
 \item Le cifre da 0 a 9, ce ne sono alcune che sono più frequenti di altre?
 \item C'è qualche sequenza di interi che appaiono frequentemente?
\end{itemize}

In realtà \textbf{sembra} che la sequenza del pi greco sia veramente casuale (ancora non dimostrato). Non è possibile prevedere la cifra successiva in base alle precedenti, non c'è periodicità.\\
Useremo la sequenza delle cifre del pi greco per generare interi casuali:

\begin{itemize}
 \item Come prima cosa abbiamo bisogno delle prime cifre del pi greco (per esempio un milione) 
	 \begin{enumerate}
 		\item Possiamo usare qualche programma che le calcoli come PiFast in ambienti Windows e SchnellPi per Linux.
 		\item Oppure possiamo scaricare questo file dal sito di \xlogo:
		\begin{center}
			\texttt{http://downloads.tuxfamily.org/xlogo/common/millionpi.txt} 
		\end{center}
 	\end{enumerate}
	\item Per generare gli interi leggeremo la sequenza di cifre in pacchetti di 7 cifre:\\
$\underbrace{3.1415926}_{\textrm{Primo numero}}\underbrace{53589793}_{\textrm{Secondo numero}}\underbrace{23846264}_{\textrm{Terzo numero}}338327950288419716939$ ecc\\ \\
Ho rimosso la virgola ``.'' in  3.14 che avrebbe causato problemi durante l'estrazione delle cifre.
\end{itemize}

Creiamo ora una nuova procedura chiamata \texttt{CasualePi} e modifichiamo la procedura \texttt{test}.
\begin{lstlisting}
Per MCD :a :b
	Se (modulo :a :b)=0 [output :b][output MCD :b modulo :a :b] 
Fine

per test :prove
	# apriamo un flusso con id 1 verso il file  millionpi.txt
	# che deve essere nella cartella attuale
	# altrimenti usare CambiaDirectory
	ApriFlusso 1 "millionpi.txt
	# assegna la variabile line per la prima linea del file millionpi.text
	AssegnaVar "line Primo LeggiLineaDalFlusso 1
	# impostiamo la variabile contatore a 0
	AssegnaVar "counter 0
	Ripeti :prove [
	  Se 1=gcd CasualePi 7 CasualePi 7 [AssegnaVar "counter :counter+1]
	]
	# Calcoliamo la frequenza
	Assegna "f :counter/:prove
	# stampiamo l'approssimazione di pi greco
	Stampa frase [ approssimazione di pi greco:] RadQ (6/:f)
	ChiudiFlusso 1
fine

per CasualePi :n
	AssegnaVarLocale "number "
	Ripeti :n [
	Se 0=count :line [AssegnaVar "line Primo LeggiLineaDalFlusso 1]
	# imposta la variabile char al primo carattere della linea
	AssegnaVar "char Primo :line
	# quindi rimuove il primo carattere dalla linea
	AssegnaVar "line EccettoPrimo :line
	AssegnaVar "number Parola :number :char
	]
	Output :number
fine
\end{lstlisting}

Invochiamo la procedura come al solito:
\begin{verbatim}
test 10
approssimazione di pi greco: 3.4641016151377544 
test 100
approssimazione di pi greco: 3.1108550841912757 
test 1000
approssimazione di pi greco: 3.081180112566604 
test 10000
approssimazione di pi greco: 3.1403714651066386 
test 70000
approssimazione di pi greco: 3.1361767950325627
\end{verbatim}

Stiamo trovato la corretta approssimazione di pi greco utilizzando le sue stesse cifre.\\
E' ancora possibile migliorare il programma calcolando il tempo del calcolo. Aggiungiamo alla prima linea della procedura \texttt{text}:\\
\texttt{AssegnaVar "inizio SecondiDaAvvio}

Quindi aggiungiamo prima di chiudere il flusso:\\
\texttt{Stampa Frase [Tempo impiegato: ] SecondiDaAvvio - :inizio}
