\chapter{Eseguire \xlogo\ sul web}
\section{Il problema}

Gestisci un sito web. Su questo sito, parli di \xlogo\ e vuoi fornire alcuni dei programmi che hai creato con \xlogo. È possibile distribuire i file logo in formato \texttt (.LGO), ma sarebbe meglio se l'utente potesse lanciare \xlogo\ on line e direttamente verificare il tuo programma.\\

Per avviare \xlogo\ da un sito web, useremo la tecnologia \textsc(Java Web Start). In realtà, abbiamo solo bisogno di mettere sul nostro sito un link verso un file con estensione \texttt(.Jnlp). Si avvierà \xlogo\ on-line.

\section{Creare il file \texttt{.jnlp}}
Questo è un esempio di questo tipo di file. Nei fatti, il successivo esempio è quello usato nel sito in francese nella sezione ``exemples''. Questo file permette di caricare il programma che carica un dado nella sezione 3D. La spiegazione del contenuto del file verrà data dopo il codice.

\begin{lstlisting}[language=XML, numbers=left, numberstyle=\tiny]
 <?xml version="1.0" encoding="utf-8"?>
<jnlp spec="1.5+" codebase="http://downloads.tuxfamily.org/xlogo/common/webstart">
<information>
  <title>XLogo</title>
  <vendor>xlogo.tuxfamily.org</vendor>
  <homepage href="http://xlogo.tuxfamily.org"/>
  <description>Logo Programming Language</description>
  <offline-allowed/>
</information>

<security>
	<all-permissions/>
</security>

<resources>
  <j2se version="1.4+"/>
  <jar href="xlogo.jar"/>
</resources>

<application-desc main-class="Launcher">
  <argument>-lang</argument>
  <argument>fr</argument>
  <argument>-a</argument>
  <argument>http://xlogo.tuxfamily.org/fr/html/examples-fr/3d/de.lgo</argument>
</application-desc>
</jnlp>

\end{lstlisting}

Questo file è scritto in formato XML. Le parti più importanti sono queste quattro linee:
\begin{lstlisting}[language=XML,numbers=left, numberstyle=\tiny,firstnumber=21]
  <argument>-lang</argument>
  <argument>fr</argument>
  <argument>-a</argument>
  <argument>http://xlogo.tuxfamily.org/fr/html/examples-fr/3d/de.lgo</argument>
\end{lstlisting}

Queste righe specificano i parametri per \xlogo\ all'avvio.

\begin{itemize}
	\item Le righe 21 e 22 forzano la lingua francese.
	\item La riga 24 indica l'indirizzo del file da caricare.
	\item La riga 23 indica che il comando principale verrà eseguito automaticamente all'avvio di \xlogo.
\end{itemize}
\vspace{0.5cm}
\textbf{Un ultimo suggerimento}: Poiché il server di Tuxfamily non può accettare tutte le connessioni è meglio mettere il file \texttt{xlogo.jar} sul tuo sito. Per collegare questo file con il file \texttt{.jnlp}, devi solo modificare l'indirizzo nella riga 2 dopo \texttt{codebase=}.