\chapter{FAQ e trucchi}

\section{Sebbene cancello una procedura dall'editor continua a ritornare}
Al momento di chiuderlo, l'editor salva o aggiorna qualsiasi cosa l'editor contiene. Per cancellare una procedura in \xlogo\ devi usare la procedura \texttt{CancellaProcedura, CancProc}. Per esempio \texttt{CancProc \textquotedbl toto} cancella la procedura \texttt{toto}.


\section{Sto usando la versione in Esperanto ma non riesco a scrivere i caratteri speciali}
Quando scrivi nella linea di comando o nell'editor cliccando il tasto destro del mouse una finestra appare. In questo menu puoi trovare le tradizionali funzioni di editing (copia, incolla, taglia) e i caratteri speciali dell'esperanto se hai selezionato questo linguaggio.


\section{Nel tab Suono della finestra di dialogo delle Preferenze non trovo alcuno strumento}
Effettivamente qualche volta l'elenco degli strumenti non appare. Consulta questa pagina: 
\begin{center}
	\texttt{http://java.sun.com/products/java-media/sound/soundbanks.html}
\end{center}
e seguine le istruzioni.


\section{Come riscrivere velocemente un comando usato in precedenza?}
\begin{itemize}
	\item Primo metodo: con il mouse, clicca sulla linea nell'area dello storico dei comandi, riapparirà immediatamente sulla linea di comando.
	\item Secondo metodo: con la tastiera le frecce Sù e Giù permettono la navigazione dello storico dei comandi.
\end{itemize}


\section{Come posso aiutare?}
\begin{itemize}
	\item Riportando un errore in \xlogo\ all'autore. Meglio se sei in grado di riprodurlo sistematicamente.
	\item Qualsiasi suggerimento per migliorare il programma è benvenuto.
	\item Aiutando nel tradurre le primitive, il manuale, il sito web.
	\item Un piccolo supporto morale è sempre benvenuto.
\end{itemize}