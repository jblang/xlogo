\chapter{Installare \xlogo}
\noindent \begin{itemize}
 \item Prima di tutto occorre installare la Java Runtime Environment (JRE) sul proprio computer. Vai a questa pagina:
\begin{center}
\texttt{ http://java.sun.com/javase/downloads/index.jsp}
\end{center}
Scarica la JRE corrispondente al proprio sistema operativo (Windows, Linux, MacOS), ed installala. 
\item Quindi occorre scaricare il file \texttt{xlogo.jar} dal seguente indirizzo: 
\begin{center}
	\texttt{http://xlogo.tuxfamily.org/common/xlogo.jar}
\end{center}
In alternativa si può andare sul sito web di \xlogo\, all'indirizzo \texttt{http://xlogo.tuxfamily.org}, scegliere una lingua e quindi cliccare sulla voce di menu Download.
\end{itemize}
\section{Configurazione di \xlogo}
\subsection{Ambiente Linux}
\textbf{Nota:} \xlogo\ è già incluso nella distribuzione OpenSuse.\\
In Ubuntu 8.04:
\begin{enumerate}
 \item Per installare la JRE:
\begin{itemize}
 \item Sistema -> Amministrazione -> Gestore Pacchetti Synaptic
 \item Installare il pacchetto \texttt{sun-java6-jre}
\end{itemize}
 \item  Per aprire il file \texttt{xlogo.jar}:
\begin{itemize}
 \item Clicca il tasto destro del mouse su \texttt{xlogo.jar}, Proprietà
 \item Tab ``Apri con'': Scegli Sun Java 6 Runtime 
\end{itemize}
 \item Per associare l'estensione \texttt{lgo} a \xlogo:
\begin{itemize}
 \item Clicca il tasto destro del mouse su \texttt{xlogo.jar}, Proprietà
 \item Tab ``Apri con'' 
 \item Bottone ``Aggiungi''
 \item Campo ``Usare un comando personalizzato'', digita (sostituendo il percorso verso \xlogo\ a ``percorso\_a\_''):
\begin{center}
\texttt{java -jar percorso\_a\_xlogo.jar} 
\end{center}
\end{itemize}
\end{enumerate}

\subsection{Ambiente Windows}
In teoria, se clicchi sull'icona di \xlogo\ il programma dovrebbe partire.
Se \xlogo\ parte correttamente, salta il resto del capitolo. Se invece viene lanciata un'altra applicazione (per esempio winzip) questo succede poiché i file .jar vengono interpretati come file .zip (ossia compressi) e viene eseguito il programma per la loro decompressione. Occorre quindi deattivare l'associazione di quel programma con i file .jar. Per far questo segui i seguenti passi per Windows XP (alcuni percorsi potrebbero essere diversi a seconda della versione di Windows in funzione, dovrai modificarli appropriatamente):

\begin{enumerate}
\item Avvio -> Pannello di controllo -> Passa alla modalità classica  -> Opzioni cartella
\item Clicca sul Tab ``Tipi di file'' (il terzo Tab)
\item Cerca nella lista dei file registrati, quelli connessi con i file jar (file jar, file jar eseguibili, archivi jar, ecc)
\item Clicca il tipo di file, quindi su Avanzate
\item Appare una nuova finestra, clicca su Sfoglia...
\item Naviga verso javaw.exe che di solito è posto presso:
\begin{center}
\texttt{c:\textbackslash{}Programmi \textbackslash{}java\textbackslash{}j2re1.4.1\textbackslash{}bin\textbackslash{}javaw.exe}
\end{center}
\item Il percorso {}``c:\textbackslash{}Programmi \textbackslash{}java\textbackslash{}j2re1.4.1\textbackslash{}bin\textbackslash{}javaw.exe'' appare quindi nel campo \textit{Applicazione utilizzata per eseguire l'azione:}. Occorre aggiungere delle informazioni alla sua fine così che si legga:
\begin{center}
\texttt{ "c:\textbackslash{}Program Files\textbackslash{}java\textbackslash{}j2re1.4.1\textbackslash{}bin\textbackslash{}javaw.exe" -jar {}"\%1" \%{*}}
\end{center}
(nota che c'è uno spazio su entrambi i lati di -jar).
\item Infine, chiudi tutte le finestre di dialogo. Ora tutto ciò che rimare da fare è cliccare sull'icona di \xlogo\ per lanciarlo!
\end{enumerate}
Se ancora \xlogo\ non parte, c'è una seconda possibilità. Apri una finestra MSDOS (su XP: Avvio -> Tutti i programmi -> Accessori -> Prompt dei comandi), quindi digitare il seguente comando (sostituendo il percorso verso \xlogo\ a ``percorso\_a\_''):
\begin{center}
\begin{verbatim}
java -jar percorso_a_XLogo

Per esempio: java -jar c:\windows\office\xlogo.jar

\end{verbatim}
(se \texttt{xlogo.jar} è posto in questa cartella).

\end{center}

Se digitare questo comando ogni volta è troppo lungo, crea un file di nome (per esempio) xlogo.bat ed inseriscici il comando stesso. Puoi quindi cliccare su questo file per lanciare \xlogo.

\subsubsection*{Associare i file con estensione .lgo a XLogo}
I file con estensione .lgo non sono di solito riconosciuti dal sistema operativo e quando si clicca su uno di loro una finestra di dialogo appare richiedendo di selezionare un'applicazione per aprirli. Selezionare \texttt{Altro} e quindi fornire il percorso a \texttt{javaw.exe} \begin{center}
Di solito è: \texttt{C:\textbackslash{}Programmi \textbackslash{}java\textbackslash{}j2re1.4.1\textbackslash{}bin\textbackslash{}javaw.exe}
\end{center}  
Viene quindi chiesto di inserire un nome per designare i file con l'estensione \texttt{.lgo}.\\
Per esempio: \texttt{File XLogo}\\
Per impostarlo come default in Windows XP, segui i seguenti passi:

\begin{enumerate}
\item Avvio -> Pannello di controlo -> Passa alla modalità classica  -> Opzioni cartella
\item Clicca sul Tab ``Tipi di file'' (il terzo Tab)
\item Cerca nella lista dei file registrati, quelli connessi con i file jar (file jar, file jar eseguibili, archivi jar, ecc)
\item Clicca il tipo di file, quindi su Nuovo
\item Digita l'estensione .lgo nel riquadro Estensione File e clicca OK
\item Clicca nella riga LGO appena aggiunta nell'elenco nei tipi di file registrati e clicca Avanzate
\item Apapre una nuova finestra, clicca su Nuovo
\item Sotto Azioni, inserisci ``apri" e quindi clicca su Sfoglia... naviga perso javaw.exe che, usualmente è in
\begin{center}
\texttt{c:\textbackslash{}Programmi \textbackslash{}java\textbackslash{}j2re1.4.1\textbackslash{}bin\textbackslash{}javaw.exe}
\end{center}
\item Clicca su Apri per aggiungere il percorso al riquadro Azioni della fienstra di dialogo Modifica Tipo di file.
\item Clicca su apri quindi su modifica
\item Il persorso {}``c:\textbackslash{}Programmi \textbackslash{}java\textbackslash{}j2re1.4.1\textbackslash{}bin\textbackslash{}javaw.exe'' appare quindi nel campo \textit{Applicazione utilizzata per eseguire l'azione:}. Occorre aggiungere delle informazioni alla sua fine così che si legga:
\begin{center}
\texttt{\textquotedbl c:\textbackslash{}Program Files\textbackslash{}java\textbackslash{}j2re1.4.1\textbackslash{}bin\textbackslash{}javaw.exe\textquotedbl  -jar {}"\%1" \%{*}}
\end{center}
\item Infine, chiudi tutte le finestre di dialogo. Ora tutto ciò che rimare da fare è cliccare sull'icona d el file lgo per lanciare \xlogo\r!
\end{enumerate}

\section{Aggiornamenti \xlogo}
\begin{center}
\includegraphics{pics/rss.png} \hspace{1cm} \texttt{http://xlogo.tuxfamily.org/rss.xml}
\end{center}
Per aggiornare \xlogo, occorre semplicemente rimpiazzare il file \texttt{xlogo.jar} con la nuova versione. 
Se vuoi ricevere un avviso quando viene pubblicata una nuova versione puoi abbonarti al feed RSS di \xlogo. L'indirizzo è 
\begin{center}
 \texttt{http://xlogo.tuxfamily.org/rss.xml}
\end{center}
Molti programmi possono gestire gli abbonamenti RSS. Se non ti senti a tuo agio con questa tecnica la scelta più facile è usare Mozilla Thunderbird:
\begin{itemize}
 \item Menu Modifica - Impostazioni Account
 \item Bottone ``Aggiungi account''
 \item ``News RSS \& Blog'', Next
 \item Nome Account: ``Feed RSS'' per esempio
 \item Bottoni ``Continua'' e ``Fine''
 \item Nella finestra principale ``Impostazioni Account'', Seleziona ``Feed RSS'' nel menu di sinistra e clicca sul bottone ``Gestione Sottoscrizioni''.
 \item Bottone ``Aggiungi''
	\begin{itemize}
 	\item URL del Feed: \texttt{http://xlogo.tuxfamily.org/rss.xml}
	\item Seleziona l'oggetto ``Mostra il sommario dell'articolo invece di caricare la pagina web''
	\end{itemize}
\end{itemize}
\vspace*{0.2cm}
È finita, con il bottone ``Ricevi posta'', puoi ricevere le news \xlogo\ insieme alle email.
\section{Disinstallazione}\label{fichier_perso}
Per disinstallare \xlogo, tutto ciò che occore fare è cancellare il file \texttt{xlogo.jar} ed il file di configurazione \texttt{.xlogo}\label{file_perso}, che è posto nella cartella home dell'utente (\texttt{/home/account} per gli utenti Linux, o \texttt{c:\textbackslash windows\textbackslash.xlogo} per gli utenti Windows).
