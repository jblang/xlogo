\chapter{Convenzioni adottate da \xlogo}

Questo capitolo espone alcuni punti chiave circa il linguaggio LOGO e circa \xlogo\ in modo specifico.


\section{I comandi e loro elaborazione}

Il linguaggio LOGO permette di invocare alcuni eventi tramite comandi interni. Questi comandi sono chiamati \textit{primitive}. Ciascuna primitiva può accettare uno o più parametri che vengono chiamati \textit{argomenti}. Per esempio, la primitiva \texttt{PulisciSchermo} non accetta alcun argomento mentre la primitiva \texttt{Somma} accetta due argomenti.\\
\texttt{Stampa Somma 2 3} restituirà 5.\\
\\
Gli argomenti LOGO sono di tre tipi:

\begin{itemize}
	\item \textbf{Numeri}: alcune primitive si aspettano un numero come argomenti: \texttt{Avanti 100} ne è un esempio. 
	\item \textbf{Parole}: le parole sono contrassegnate dai doppi apici (\textquotedbl ) all'inizio del nome della parola stessa. Un esempio di una primitiva che accetta una parola come argomento è \texttt{Stampa}. 
	\begin{center}
		\texttt{Stampa \textquotedbl ciao}
	\end{center}
	Questo comando visualizza \texttt{ciao}. Se viene dimenticato il carattere \textquotedbl\ l'interprete logo restituirà un messaggio di errore. Per l'interprete LOGO \texttt{ciao} non rappresenta niente poiché non è un numero, una parola, un elenco o una procedura già definita.
	\item \textbf{Elenchi}: gli elenchi sono definiti entro parentesi quadre.
\end{itemize}
\textbf{Nota}: i Numeri sono trattati a volte come valori numerici (per esempio \texttt{Avanti 100}), altre volte come parole (per esempio \texttt{Stampa Primo 12} scrive \texttt{1}). \\

\subsection{Le primitive generali}

Molte primitive possiedono una forma generale ossia possono essere usate con un numero di argomenti indefinito. Queste primitive sono:
\begin{center}
	\begin{tabular}{cccc}
		\texttt{Stampa} & \texttt{Somma}&\texttt{Prodotto} &\texttt{o}\\
		\hline
		\texttt{e}&\texttt{Elenco}&\texttt{Frase}& \texttt{Parola}\\
	\end{tabular} 
\end{center}
Per notificare l'interprete LOGO che queste primitive saranno usate nella loro forma generale occorre inscrivere il comando fra parentesi tonde, come nell'esempio seguente:
\begin{lstlisting}
Stampa (Somma 1 2 3 4 5)
# 15

Stampa (Elenco [a b] 1 [c d])
# [a b] 1 [c d] 

Se (e 1=1 2=2 8=5+3) [Avanti 100 RuotaDestra 90]
\end{lstlisting}

\section{Le procedure e le variabili}

Oltre alle primitive è possibile definire comandi personalizzati. Questi comandi sono chiamati \textit{procedure}. Le procedure sono definite mediante le primitive \texttt{Per \textellipsis Fine}. Il blocco di comandi che la procedura eseguirà viene posto all'interno delle due precedenti primitive. Esse possono essere create utilizzando l'editor interno di \xlogo. Ecco un breve esempio:
\begin{lstlisting}
Per quadrato
	Ripeti 4 [Avanti 100 RuotaDestra 90]
Fine
\end{lstlisting}

Come le primitive anche le procedure possono trarre vantaggio degli argomenti. Per passare argomenti alle procedure si utilizzano le variabili. Una variabile è una parola alla quale si può associare (assegnare in termini informatici) un valore. Le variabili sono quindi una sorta di contenitori che possono essere riempiti di valori a nostro piacimento. Ecco un semplice esempio:
\begin{lstlisting}
Per totale :a :b
	Stampa Somma :a :b
Fine
\end{lstlisting} 

Invocando \texttt{totale 2 3} otterremo 5 come risultato.


\section{Il carattere speciale \textbackslash}
Il carattere speciale  \textbackslash \ (barra rovesciata) permette la creazione di parole contenenti simboli vuoti o di particolare significato come l'andare a capo. Se \textbackslash n è usato la frase salta alla linea successiva e \textbackslash\textvisiblespace\ seguito da uno spazio permette di inserire uno spazio in una parola. Per esempio:
\begin{lstlisting}
	Stampa "xlogo\ xlogo
	# xlogo xlogo
	Stampa "xlogo\nxlogo
	# xlogo
	# xlogo
\end{lstlisting}

Per inserire la barra rovesciata in una parola occorre scrivere \textbackslash\textbackslash.\\
Allo stesso modo, per includere quei caratteri a cui \xlogo\ assegna particolari significati ( ( ) [ ] \# ) in una parola occorre prefissarli con la barra rovesciata.  \\
\textbf{Tutti i simboli preceduti da \textbackslash \ sono ignorati. Questo è particolarmente importante nello scrivere i nomi dei file.}. Per esempio per impostare il percorso attuale a \texttt{c:\textbackslash Miei Documenti}:
\begin{lstlisting}
	ImpDir "c:\\Miei\ Documenti.
\end{lstlisting}

Da notare l'uso di \textbackslash\textvisiblespace \ per notificare all'interprete LOGO dell'esistenza dello spazio fra Miei e Documenti. Se si omette la doppia barra rovesciata il percorso diventa  \texttt{c:Miei Documenti} e l'interprete restituirà un messaggio di errore circa l'inesistenza di tale percorso.


\section{Maiuscole e minuscole}

\xlogo\ non fa differenza tra maiuscole e minuscole nei nomi delle procedure e delle primitive. Quindi la procedura \texttt{quadrato} definita precedentemente può essere invocata come \texttt{QUADRATO}, \texttt{Quadrato}, \texttt{qUadrato} e così via, l'interprete LOGO la eseguirà in ogni caso. Al contrario \xlogo\ differenzia le maiuscole dalle minuscole nel caso degli elenchi e delle parole, per esempio:
\begin{lstlisting}
Stampa "Ciao
# Ciao (la maiuscola iniziale viene conservata)
\end{lstlisting}


\section{Gli operatori e la sintassi}

Ci sono due modi per scrivere taluni comandi. Per esempio per sommare 4 e 7 si può usare la primitiva \texttt{Somma} che richiede due argomenti: \texttt{Somma 4 7}, o si può usare l'operatore ``+'': \texttt{4+7}. Entrambi i modi hanno il medesimo effetto. La seguente tabella illustra la relazione tra operatori e primitive:
\begin{center}
	\begin{tabular}{|c|c|c|c|}
		\hline 
		\texttt{Somma}&		\texttt{Differenza}&		\texttt{Prodotto}&		\texttt{Quoziente}\\
		\hline
		+ &		- &		{*} &		/ \\
		\hline
		\texttt{o}&		\texttt{e}&		\texttt{uguale?}&\\
		\hline
		| (ALT GR+6) &		\&&		=&\\
		\hline
	\end{tabular}
\end{center}
\vspace{0.25cm}
Ci sono due altri operatori che non sono associati ad alcuna primitiva:
\begin{itemize}
	\item l'operatore ``Minore o uguale'': \texttt{<=}
	\item l'operatore ``Maggiore o uguale'': \texttt{>=}
\end{itemize}

Nota: I due operatori | e \& sono specifici di \xlogo. Non esistono nelle versioni tradizionali di LOGO. Qualche esempio di impiego:
\begin{lstlisting}
Stampa 3+4=7-1 			# falso
Stampa 3=4 | 7<=49/7	# vero
Stampa 3=4 & 7=49/7		# falso
\end{lstlisting}
