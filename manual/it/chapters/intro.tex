\chapter{Introduzione}
\logo\ \`e un linguaggio di programmazione sviluppato negli anni 60 da Seymour Papert. Papert \`e stato lo sviluppatore una teoria sull'apprendimento molto originale ed influente chiamata ``costruzionismo'' che si pu\`o riassumere con l'espressione ``imparando facendo''.\\ \\
\logo\ \`e un linguaggio molto appropriato per sviluppare le capacit\`a matematiche e logiche. \`E un eccellente linguaggio per cominciare a programmare, infatti insegna le fondamenta di cose come i cicli, i test, le procedure ecc. L'utente muove un oggetto chiamato ``tartaruga'' per tutto lo schermo, utilizzando semplici comandi come avanti, indietro, destra e cos\`i via. Muovendosi la tartaruga lascia una traccia dietro di s\`e rendendo possibile la creazione di disegni. Il fatto che l'utente possa impartire alla tartaruga ordini in un linguaggio molto naturale rende  \logo\ molto semplice da imparare. Un uso pi\`u avanzato \`e comunque possibile essendo possibile operare su liste, parole o file.\\ \\
\xlogo\ \`e un interprete \logo\, cioè le istruzioni dell'utente vengono eseguite direttamente. L'utente può osservare gli errori nel programma immediatamente a schermo. Questo approccio grafico molto intuito rende \logo\ un linguaggio ideale per i principianti, specialmente per i bambini!\\ \\
L'indirizzo principale del sito web di \xlogo\ è
\begin{center}
\texttt{http://xlogo.tuxfamily.org/}
\end{center}
Lì è possibile ottenere sia il software sia la documentazione. Sul sito si trovano anche molti esempi creati con \xlogo\ così da potersi fare un'opinione sulle capacità di \xlogo.\\ \\
\xlogo\ supporta ora undici linguaggi (arabico, asturiano, inglese, italiano, esperanto, francese, galiziano, greco, tedesco, portoghese, e spagnolo). \xlogo\ è scritto in \textsc{Java} - un linguaggio di programmazione con il vantaggio di essere multi-piattaforma - rendendolo utilizzabile su macchine Linux, Windows e MacOS senza alcun problema.\\ \\
\xlogo\ è posto sotto licenza GPL. \`E quindi software libero ed i suoi utenti hanno quattro libertà:
\begin{itemize}
\item Libertà 1: La libertà di eseguire il programma per qualsiasi scopo.
\item Libertà 2: La libertà di studiare e modificare il programma.
\item Libertà 3: La libertà di copiare il programma così da aiutare il prossimo.
\item Libertà 4: Libertà di migliorare il programma e di distribuirne pubblicamente i miglioramenti, in modo tale che tutta la comunità ne tragga beneficio.
\end{itemize}
\vspace{0.3cm}
\noindent \textbf{Organizzazione del manuale:}\\ \\
Il manuale ti aiuterà a scoprire \xlogo:
\begin{itemize}
 \item Nella prima parte vengono spiegati i diversi menu e le opzioni dell'interfaccia.
 \item Successivamente, alcuni capitoli presentano le istruzioni più importanti per cominciare ad usare \xlogo. Le prime istruzioni sono molto semplici, poi la complessità cresce. A volte il capitolo si chiude con qualche esercizio la cui soluzione si trova nell'appendice D.
\item Infine una serie di temi approfonditi si presentano per gli utenti esperti.
\item Nell'appendice A sono elencate tutte le primitive di  \xlogo, ciascuna dettagliatamente descritta.
\end{itemize}
\vspace{0.5cm}
Questo manuale è disponibile in diversi formati:
\begin{itemize}
 \item \textsc{PDF}: http://downloads.tuxfamily.org/xlogo/downloads-it/manual-it.pdf
 \item \textsc{HTML compresso}: http://downloads.tuxfamily.org/xlogo/downloads-it/manual-html-it.zip
 \item \LaTeXe: Sorgente: http://downloads.tuxfamily.org/xlogo/downloads-fr/manual-src-it.zip
 \item \textsc{JavaHelp}: Menu Aiuto-Manuale Online in \xlogo
\end{itemize}
