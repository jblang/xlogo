\chapter{Launching \xlogo\ with command line}
\noindent
To execute \xlogo, here is the syntax of the command:
\begin{center}
 \texttt{java -jar xlogo.jar [-a] [-lang en] [-memory 64] [file1.lgo file2.lgo ...] } 
\end{center}
List of available options: \\
\begin{itemize}
 \item Attribute \texttt{-lang}: this attribute specifies a language for \xlogo. This parameter overwrites the one from the config file called  \texttt{.xlogo}. Have a look at the following table which shows all available languages: \\
\begin{center}
\begin{tabular}{|c|c|c|c|c|c|c|c|c|}
\hline
French & English & Spanish & german & Arabic & Portuguese & Espéranto & Galician & Greek\\
\hline
fr & en & es & de & ar & pt & eo & gla el\\
\hline
\end{tabular}
\end{center}
\vspace{0.5cm}
\item Attribute \texttt{-a}: this attribute indicates execution of the main command, contained in the loaded files on startup, after \xlogo's window has opened.\\
\item Attribute \texttt{-memory}: this attribute changes the corresponding memory space allocated to \xlogo.\\
\item file1.lgo, file2.lgo ...: these files in format \texttt{.lgo} are loaded on \xlogo\ startup. These files could be local or distant. Hence, you can specify a local address or a web address.\\
\item Attribute \texttt{-tcp\_port}: this attribute allows to modify the default TCP port used for networking (See p.\pageref{network}). By default, its value is 1948.
\end{itemize}
\vspace{0.5cm}
A few examples: 
\begin{itemize}
 \item \texttt{java -jar xlogo.jar -lang es prog.lgo}:\\
 Files \texttt{xlogo.jar} and \texttt{prog.lgo} are in the current directory. This command executes \xlogo, with language configured to spanish. Then, it loads the file \texttt{prog.lgo} (Thus, this file is written in spanish...)\\
 \item \texttt{java -jar xlogo.jar -a -lang en http://xlogo.tuxfamily.org/prog.lgo}:\\
 This command executes \xlogo\ in english. It loads the file \texttt{http://xlogo.tuxfamily.org/prog.lgo}. Finally, the main command from this file is executed on startup.\\
\end{itemize}