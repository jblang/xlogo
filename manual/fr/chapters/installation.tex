\chapter{Installation de \xlogo}
\noindent 
\begin{itemize}
 \item Premièrement, il vous faut installer un environnement d'exécution JAVA sur votre ordinateur. Rendez-vous sur cette page:
\begin{center}
 \texttt{http://java.sun.com/javase/downloads/index.jsp}
\end{center}
Télécharger le JRE (Java Runtime Environment) correspondant à votre systèmee d’exploitation (windows, Linux ...) puis installez-le.
\item Deuxièmement, il faut télécharger le fichier \texttt{xlogo.jar} situé à l'adresse:
\begin{center}
 \texttt{http://downloads.tuxfamily.org/xlogo/common/xlogo.jar}
\end{center}
Sinon, plus simplement, rendez-vous sur le site de \xlogo, à l'adresse \texttt{http://xlogo.tuxfamily.org} puis choisir le français et le menu téléchargement.
\end{itemize}
\section{Configuration de \xlogo}
\subsection{Environnement Linux}
Sous Ubuntu 8.04:
\begin{enumerate}
 \item Pour installer JAVA:
\begin{itemize}
 \item Système -> Administration -> Gestionnaire de paquets Synaptic
 \item Installer le paquet \texttt{sun-java6-jre}
\end{itemize}
 \item  Pour ouvrir le fichier \texttt{xlogo.jar} par double-clic:
\begin{itemize}
 \item Clic droit sur \texttt{xlogo.jar}, Propriétés
 \item Onglet \og Ouvrir avec\fg: Choisir Sun Java Runtime 
\end{itemize}
 \item Asscocier les extensions \texttt{lgo} à \xlogo:
\begin{itemize}
 \item Clic droit sur \texttt{xlogo.jar}, Propriétés
 \item Onglet \og Ouvrir avec\fg:
 \item Bouton \og Ajouter \fg
 \item Dans \og Utiliser une comande personnalisée:\fg, taper:
\begin{center}
\texttt{java -jar chemin\_vers\_xlogo.jar} 
\end{center}
\end{itemize}
\end{enumerate}
\textbf{Remarque:} \xlogo\ est inclus dans la distribution OpenSuse.
\subsection{Environnement Windows}
    En principe, si vous double-cliquez sur l’icône de \xlogo, le logiciel doit se lancer. Si c’est le cas,
passer au paragraphe suivant. Sinon, c'est qu'un autre programme prend en charge les fichiers d'extension \og\texttt{jar}\fg\ (Souvent, des programmes de décompression, style WinZip et autres).\\ \\
 Voilà comment associer le programme \og\texttt{java}\fg\ aux fichiers d'exetnsion \og \texttt{jar}\fg. (Certains chemins peuvent différés selon que vous possédiez Windows 98, 2000, XP ...)
\begin{enumerate}
 \item Démarrer–> Paramètres —> Options des dossiers ... 
 \item Cliquez ensuite sur l’onglet \og Types de fichiers\fg\ (le 3\textsuperscript{ème}). 
 \item Repérez alors dans la liste des options proposées celles se rapportant au fichiers JAR (Fichiers JAR, Fichiers exécutables JAR, archive JAR ...) 
 \item Sélectionner ce type de fichiers et cliquez sur \og Modifier...\fg 
 \item Une nouvelle fenêtre apparaît alors choisissez \og Modifier... \fg 
 \item Séectionner alors \og Parcourir...\fg
 \item Il faut indiquer le chemni vers \texttt{javaw.exe}, en principe
\begin{center}
 \texttt{c:\textbackslash Program Files\textbackslash java\textbackslash j2re1.4.1\textbackslash bin\textbackslash javaw.exe}
\end{center}
 \item Une fois ceci fait, il apparaît dans le champ Application utilisée pour réaliser l’action :
\begin{center}
 \texttt{c:\textbackslash Program Files\textbackslash java\textbackslash j2re1.4.1\textbackslash bin\textbackslash javaw.exe}
\end{center}
Il faut alors rajouter au bout :
\begin{center}
\texttt{"c:\textbackslash Program Files\textbackslash java\textbackslash j2re1.4.1\textbackslash bin\textbackslash javaw.exe" -jar "\%1" \%*}
\end{center} 
(Attention il faut un espace de chaque côté de -jar)
\item Ensuite, il n'y a plus qu'à refermer toutes les fenêtres puis double-cliquer sur l'icône de \xlogo. 
\end{enumerate}
 Si ca ne fonctionne toujours pas, il y a une deuxième possiblité: Vous ouvrez une console MSDOS (Démarrer –> Programmes –> Commandes MSDOS ou Démarrer—> Programmes—-> Accessoires—-> Invite MSDOS) puis vous tapez l’instruction suivante :
\begin{center}
 \texttt{java -jar le\_chemin\_ou\_se\_trouve\_le\_fichier}
\end{center}
Par exemple : \texttt{java -jar c:\textbackslash xlogo\textbackslash xlogo.jar}\\ \\
    Si cela vous ennuie de systématiquement devoir taper cette instruction. Taper là dans un fichier texte et
enregistrer le par exemple sous le nom de \texttt{xlogo.bat}. Il ne reste plus qu'à double cliquer sur \texttt{xlogo.bat} pour lancer \xlogo.

\subsubsection*{Associer les fichiers d'extension \texttt{lgo} avec \xlogo}
Je n'ai pas réussi à configurer cela sous Vista. (Je n'ai pas trop cherché non plus... avis aux amateurs! Merci de me communiquer la solution)\\ \\
     En principe, les fichiers d'extension \texttt{.lgo} ne sont pas reconnus par votre ordinateur, lorsque vous double-cliquez
dessus, une boîte de dialogue apparaît vous demandant quelle application il faut utiliser pour ouvrir le fichier.
\begin{itemize}
 \item  Indiquer \og Autre \fg puis indiquer le chemin vers l’application \texttt{javaw.exe}
\begin{center}
 Généralement, \texttt{c:\textbackslash Program Files\textbackslash java\textbackslash j2re1.4.1\textbackslash bin\textbackslash javaw.exe}
\end{center}
\item Doner un nom pour désigner les fichiers d'extension \texttt{lgo}.\\
Par exemple: Fichiers Logo
\item Démarrer -> Paramèters -> Options des dossiers
\item Onglet \og Type de Fichiers \fg
\item Repérer dans la liste les fichiers \texttt{lgo}
\item Sélectionner ce type de fichiers puis cliquer sur \og Modifier\fg
\item Une nouvelle fenêtre apparaît, Encore \og Modifier\fg
\item Dans le champ \og Application utilisée pour réaliser l'action\fg,
\begin{center}
\texttt{"c:\textbackslash Program Files\textbackslash java\textbackslash j2re1.4.1\textbackslash bin\textbackslash javaw.exe" -jar xlogo.jar "\%1" \%*}
\end{center} 
\item Reste à fermer les fenêtres
\end{itemize}
\section{Mises à jour}
\begin{center}
\includegraphics{images/rss.png} \hspace{1cm} \texttt{http://xlogo.tuxfamily.org/rss.xml}
\end{center}
Pour mettre à jour \xlogo, il suffit de remplacer le fichier \texttt{xlogo.jar} par sa nouvelle version. 
Si vous souhaitez être prévenu de la parution de chaque nouvelle version, ou de chaque amélioration, il est possible de s'abonner au fil RSS de \xlogo. L'adresse du fill RSS est:
\begin{center}
 \texttt{http://xlogo.tuxfamily.org/rss.xml}
\end{center}
Il existe plusieurs logiciels permettant de gérer les fils RSS, si vous n'êtes pas coutumier de cette technique, le plus simple est d'utiliser Mozilla Thunderbird:
\begin{itemize}
 \item Menu Edition - Paramètre des comptes
 \item Bouton \og Ajouter un compte\fg
 \item \og Nouvelles RSS et Blogs\fg
 \item Nom du compte: \og Fils RSS\fg\ par exemple
 \item Boutons \og Suivant\fg\ et \og Terminer\fg
 \item Dans la fenêtre \og Paramètre des comptes\fg, sélectionner alors \og Fils RSS\fg\ dans le menu de gauche puis cliquer sur le bouton \og Gérer les abonnements\fg.
 \item Bouton \og Ajouter\fg
	\begin{itemize}
 	\item URL du fil: \texttt{http://xlogo.tuxfamily.org/rss.xml}
	\item  Cocher la case \og Aficher le résumé de l'article plutôt que de télécharger la page Web\fg
	\end{itemize}
\end{itemize}
\vspace*{0.2cm}
Voilà, avec le bouton \og Envoyer-Recevoir\fg, vous recevrez les nouvelles de \xlogo\ de la même façon que vous gérez vos mails.
\section{Désinstallation}\label{fichier_perso}
Pour désinstaller \xlogo, il suffit de supprimer le fichier \texttt{xlogo.jar} et le fichier de démarrage \texttt{.xlogo} (il est situé dans votre répertoire utilisateur c'est à dire \texttt{/home/votre\_login} pour les linuxiens ou \texttt{c:\textbackslash windows\textbackslash.xlogo}
