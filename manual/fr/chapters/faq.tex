\chapter{Foire aux questions - Astuces - trucs à connaître}
\section{J'ai beau effacer une procédure dans l'éditeur, elle réapparaît  tout le temps!}
 Lorsqu'on sort de l'éditeur, celui-ci se contente juste de sauver ou de mettre à jour le contenu de l'éditeur. Le seul moyen d'effacer une procédure dans XLogo est d'utiliser la primitive \texttt{effaceprocedure} ou \texttt{efp}. \\
 Exemple: \texttt{effaceprocedure "toto} $\longrightarrow$ efface la procédure \texttt{toto}.
 \section{J'utilise la version en espéranto mais je ne peux écrire les caractères spéciaux!}
 Lorsque vous tapez dans la ligne de commande ou l'éditeur, si vous faites un clic avec le bouton droit de la souris, apparait un menu déroulant. Dans ce menu, figurent les traditionnels fonctions d'édition (copier, couper, coller) et les caractères spéciaux de l'espéranto lorsque ce langage est sélectionné.
 \section{Dans l'onglet Son de la boîte de dialogue Préférences, aucun instrument n'est disponible.}
Parfois, la liste des instruments MIDI n'apparait pas dans \texttt{Outils/Préférences/Son} et on ne peut pas exploiter totalement les fonctionnnalités sonores de XLogo. Rendez-vous à cette adresse:
\begin{center}
 \texttt{http://java.sun.com/products/java-media/sound/soundbanks.html}
\end{center}
Télécharger une des banques sonores (soundbank) proposées : minimal, midsize ou deluxe puis décompresser-la dans \texttt{C:\textbackslash Program Files\textbackslash Java\textbackslash jre1.6.0\_05\textbackslash lib\textbackslash audio\textbackslash}.\\
\begin{itemize}
 \item le dossier \texttt{jre1.6.0\_05} correspond à votre version du JRE installée.
 \item si le dossier \texttt{audio} n'existe pas, il faudra le créer.
 \item il faudra renommer le fichier décompressé en : \texttt{soundbank.gm}
\end{itemize}
\vspace{0.2cm}
\noindent Ensuite relancez XLogo et allez donc voir dans \texttt{Outils/Préférences/Son }

\section{Comment faire pour taper rapidement une commande déjà utilisée au préalable?}
\begin{itemize}
\item Première méthode: avec la souris, cliquer dans la zone d'historique sur la ligne désirée, elle réapparaîtra immédiatement dans la ligne de commande.
\item Deuxième méthode: avec le clavier, les flèches Haut et Bas permettent de naviguer dans la liste des dernières commandes tapées(Très pratique).
\end{itemize}
 \section{Comment peut-on vous aider?}
\begin{itemize}
\item En reportant tout bug constaté. Si vous êtes capable de reproduire systématiquement un problème constaté, c'est encore mieux. 
\item Vos suggestions en vue de l'amélioration sont toujours les bienvenues.
\item En aidant aux traductions: en particulier l'anglais...
\item Un petit encouragement fait toujours du bien!
\end{itemize}
