\chapter{Corrigé des activités}
\section{Chapitre 2}
\begin{verbatim}
pour carre
repete 4[av 150  td 90]
fin

pour tri
repete 3[av 150 td 120]
fin

pour porte
repete 2[av 70 td 90 av 50 td 90]
fin

pour che
av 55 td 90 av 20 td 90 av 20
fin

pour dep1
td 90 av 50  tg 90
fin

pour dep2
tg 90 av 50 td 90 av 150 td 30
fin

pour dep3
lc td 60 av 20 tg 90 av 35 bc
fin

pour ma
carre dep1 porte dep2 tri dep3 che
fin

\end{verbatim}

\section{Chapitre 3}
\begin{verbatim}
pour supercube
ve lc fpos[ -30 150] bc fpos[-150 150]  fpos[-90 210] fpos[30 210] fpos[-30 150]
fpos[-30 -210] fpos[30 -150] fpos[30 -90] fpos[-30 -90] fpos[90 -90] fpos[90 30]  
fpos[-270 30] fpos[-270 -90] fpos[-210 -90] fpos[-210 -30] fpos[-90 -30] fpos[-90 -150]
fpos[-210 -150] fpos[-210 -30] fpos[-150 30] fpos[-30 30] fpos[-90 -30] fpos[90 150]
fpos[30 150] fpos[30 210] fpos[30 90] fpos[90 90] fpos[90 150] fpos[90 90] fpos[150 90]
fpos[150 -30] fpos[90 -90] fpos[90 30] fpos[150 90] lc fpos[-150 30] bc fpos[-150 150] 
fpos[-150 90] fpos[-210 90] fpos[-270 30] lc  fpos[-90 -150] bc fpos[-30 -90]
lc fpos[-150 -150] bc fpos[-150 -210] fpos[-30 -210]
fin
\end{verbatim}

\section{Chapitre 4}
\subsection{Le robot}
Le premier dessin est composé exclusivement de motif élémentaire à base de rectangle, carré et triangle. Voici le code associé à ce dessin:
\begin{verbatim}
pour rec :lo :la
# trace un rectangle de longueur :lo et largeur :la
repete 2[av :lo td 90 av :la td 90]
fin

pour carre :c
# trace un carre de cote :c
repete 4[av :c td 90]
fin

pour tri :c
# trace un triangle equlateral de côté :c
repete 3[av :c td 120]
fin

pour patte :c
rec 2*:c 3*:c carre 2*:c
fin

pour antenne :c
av 3*:c tg 90 av :c td 90 carre 2*:c
lc re 3 *:c td 90 av :c tg 90 bc
fin

pour robot :c
ve ct
# Le corps
rec 4*:c 28* :c
# Les pattes
td 90 av 2*:c patte :c av 4* :c patte :c av 14*:c patte :c av 4*:c patte :c
# La queue
lc tg 90 av 4* :c bc td 45 av 11*:c re 11 * :c tg 135
# le cou et la tête
av 18 *:c carre :c av 3*:c carre :c td 90 av :c tg 90 av 2*:c td 90 carre 8* :c
# Oreilles
av 4*:c tg 60 tri 3*:c lc td 150 av 8 *:c tg 90 bc tri 3*:c
# Les antennes
av 4 *:c tg 90 av 2*:c td 90 antenne :c tg 90 av 4*:c td 90 antenne :c
# les yeux 
lc re 3 *:c bc carre :c td 90 lc av 3*:c bc tg 90 carre :c
# La bouche
lc re 3*:c tg 90 av 3*:c td 90 bc rec :c 4*:c
fin
\end{verbatim}
\subsection{La grenouille}
\begin{verbatim}
pour gre :c
ve ct
av 2 *:c td 90 av 5*:c tg 90 av 4*:c tg 90 av 7 *:c td 90 av 7*:c td 90
av 21 *:c td 90 av 2*:c tg 90 av 2*:c td 90 av 9*:c td 90 av 2*:c tg 90
av 2*:c td 90 av 9*:c td 90 av 2*:c td 90 av 7*:c re 5*:c tg 90 av 4*:c 
td 90 av 4*:c re 4*:c tg 90 re 2*:c tg 90 av 5*:c tg 90 av 4*:c td 90 av 7*:c 
td 90 lc av 9*:c bc repete 4[av 2*:c td 90]
fin
\end{verbatim}

\section{Chapitre 8:}
\begin{verbatim}
pour jeu
# On initialise le ,nombre recherché et le nombre de coups
donne "nombre hasard 32
donne "compteur 0
boucle
fin

pour boucle
lis [proposez un nombre] "essai
si nombre? :essai[
  # Si la valeur rentrée est bien un nombre 
  si :nombre=:essai[ec ph ph [vous avez gagné en ] :compteur+1 [coup(s)]][
    si :essai>:nombre [ec [Plus petit]][ec [Plus grand]]
    donne "compteur :compteur+1
    boucle
  ]
]
[ecris [Vous devez rentrer un nombre valide!] boucle]
fin
\end{verbatim}