\chapter{Gesti\'on de tiempos}
   \label{Gestion-Tiempos}
   \index{Gesti\'on de tiempos}

\textsc{XLogo} dispone de varias primitivas que permiten conocer la hora y
la fecha o utilizar un cron\'ometro descendente (\'util para repetir una
tarea a intervalos fijos).
\begin{center} \begin{longtable}{|m{3cm}|m{3cm}|m{9cm}|} \hline 
   \multicolumn{1}{|c|}{\textbf{Primitivas}} &
      \multicolumn{1}{c|}{\textbf{Argumentos}} &
         \multicolumn{1}{c|}{\textbf{Uso}} \\ \endhead \hline 
   \texttt{espera}\index{espera@\texttt{espera}} &
      \texttt{n: n\'umero entero} &
        Hace una pausa en el programa, la tortuga espera
        \texttt{(n/60)} segundos.\\ \hline 
   \texttt{cron\'ometro},\index{cron\'ometro@\texttt{cron\'ometro}}
     \texttt{crono}\index{crono@\texttt{crono}} & 
        \texttt{n: n\'umero entero} &
        Inicia un conteo descendiente de \texttt{n} segundos. Para saber que
        la cuenta ha finalizado, disponemos de la primitiva
        \texttt{fincrono?} \\ \hline 
   \texttt{fincron\'ometro?},\index{fincron\'ometro?@\texttt{fincron\'ometro?}}
      \texttt{fincrono?}\index{fincrono?@\texttt{fincrono?}} & \texttt{no} &
        Devuelve \verb+"cierto+ si no hay ning\'un conteo activo.
        Devuelve \verb+"falso+ si el conteo no ha terminado.\\ \hline 
   \texttt{fecha}\index{fecha@\texttt{fecha}} & \texttt{no} &
        Devuelve una lista compuesta de 3 n\'umeros enteros que representan
        la fecha del sistema. El primero indica el d\'ia, el segundo el mes
        y el \'ultimo el a\~no. \texttt{[d\'ia mes a\~no]} \\ \hline 
   \texttt{hora}\index{hora@\texttt{hora}} & \texttt{no} &
        Devuelve una lista compuesta de 3 n\'umeros enteros que representan
        la hora del sistema. El primero representa las horas, el segundo los
        minutos y el \'ultimo los segundos. \texttt{[horas minutos segundos]}
                        \\ \hline 
   \texttt{tiempo}\index{tiempo@\texttt{tiempo}} & \texttt{no} &
        Devuelve el tiempo, en segundos, transcurrido desde el inicio de
        \textsc{XLogo}.\\ \hline
\end{longtable} \end{center}
Veamos un procedimiento de ejemplo:
\begin{verbatim}
   para reloj 
    # muestra la hora en forma numerica (actualizada cada 5 segundos)
     si fincrono? [ 
      bp ponfuente 75 ot
      haz "ho hora   
      haz "h primero :ho
      haz "m elemento 2 :ho
    # muestra dos cifras para los minutos (completando el 0)
      si :m - 10 < 0 [
       haz "m palabra 0 :m ]
      haz "s ultimo :ho
    # muestra dos cifras para los segundos
      si :s - 10 < 0 [
       haz "s palabra 0 :s ]
       rotula (palabra :h ": :m ": :s) crono 5 ]
     reloj
   fin 
\end{verbatim}
