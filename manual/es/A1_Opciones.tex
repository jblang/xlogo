\chapter{Iniciando \textsc{XLogo} desde la l\'inea de comandos}

\noindent La sintaxis del comando que ejecuta \textsc{XLogo} es:
\begin{verbatim}
  java-jar xlogo.jar [-a] [-lang en] [archivo1.lgo archivo2.lgo ...]
\end{verbatim}
Las opciones disponibles son:
\begin{itemize}
   \item El atributo \texttt{-lang}: este atributo especifica el idioma con que
      ejecutar \textsc{XLogo}.

      El par\'ametro indicado con esta opci\'on se superpone al existente en el
      fichero \texttt{.xlogo} de configuraci\'on personal. En la siguiente tabla
      se muestran los par\'ametros asociados a todos los idiomas disponibles:
      \begin{center} \begin{tabular}{|*{5}{c|}} \hline
         Franc\'es & Ingl\'es & Espa\~nol & Alem\'an & Portugu\'es \\ \hline
             \texttt{fr} & \texttt{en} & \texttt{es} & \texttt{de} & \texttt{pt} \\ \hline
              \multicolumn{5}{c}{ } \\ \hline
          \'Arabe & Esperanto & Gallego & Asturiano &  \\ \hline
             \texttt{eo} & \texttt{ar} & \texttt{gl} & \texttt{as} & \\ \hline
      \end{tabular} \end{center}
   \item Atributo \texttt{-a}: este atributo determina que se ejecute el 
      \textbf{Comando de Inicio} (ver secci\'on \ref{EditorProcedimientos}) que
      figura en el archivo cargado sin necesidad de pulsar el bot\'on de inicio
      (secci\'on \ref{La-ventana-principal}) al iniciarse \textsc{XLogo}.
    \item Atributo \texttt{-memory}: este atributo cambia el tama\~no de memoria
       reservado para \textsc{XLogo}
    \item \texttt{[archivo1.lgo archivo2.lgo ...]}: estos archivos de extensi\'on 
       \texttt{.lgo} se cargan al iniciar \textsc{XLogo}.

       Los archivos pueden ser locales (estar en el disco duro del ordenador) o
       remotos (en la \textit{web}). Por lo tanto, se puede especificar una
       direcci\'on local o una direcci\'on web.
\end{itemize}

Veamos ejemplos:
\begin{itemize}
   \item \texttt{java -jar xlogo.jar -lang es prog.lgo} 

      los archivos \texttt{xlogo.jar} y \texttt{prog.lgo} est\'an en el directorio
      actual. Este comando ejecuta \textsc{XLogo} cambiando el idioma a ``espa\~nol''
      y a continuaci\'on carga el fichero \texttt{prog.lgo}. Por tanto, es necesario
      que este archivo sea un programa \textsc{Logo} escrito en Espa\~nol.
   \item \texttt{java -jar xlogo.jar -a -lang en http://xlogo.tuxfamily.org/prog.lgo}:

      Este comando ejecuta \textsc{XLogo} en Ingl\'es, estando \texttt{xlogo.jar} en
      el directorio de trabajo, y se carga el fichero 
      \mbox{\texttt{http://xlogo.tuxfamily.org/prog.lgo}} desde la \textit{web}.

      Por \'ultimo, el Comando de Inicio de este fichero se ejecuta en el arranque.
\end{itemize}